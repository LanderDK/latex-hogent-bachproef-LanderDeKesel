%---------- Inleiding ---------------------------------------------------------


\section{Introductie}%
\label{sec:introductie}

Deze bachelorproef richt zich op authenticatie en autorisatie in applicatieontwikkeling, met een specifieke focus op het 
vereenvoudigen van deze processen voor developers. De aanleiding voor deze bachelorproef komt voort uit ervaring als IT'er en ontwikkelaar, 
waar aanvankelijk uitdagingen werden ondervonden bij het beveiligen en beschikbaar stellen van applicaties voor gebruikers zonder ongeautoriseerde toegang.
\newline
\newline
De doelgroep van deze bachelorproef bestaat uit IT professionals en ontwikkelaars die betrokken zijn bij het ontwerp en de ontwikkeling van
applicaties. De doelgroep heeft een basiskennis van applicatieontwikkeling en beveiliging, maar heeft mogelijk geen ervaring met het ontwikkelen
van een eigen authenticatie- en autorisatiesysteem en heeft dus behoefte aan een oplossing die gemakkelijk te implementeren is in hun applicaties.
\newline
\newline
De centrale probleemstelling van dit onderzoek is het identificeren van de moeilijkheden die ontwikkelaars ondervinden bij het implementeren van
een eigen authenticatie- en autorisatiesysteem in hun applicaties. De onderzoeksvragen die hieruit voortvloeien zijn:
\begin{itemize}
  \item Wat zijn de moeilijkheden die ontwikkelaars ondervinden bij het implementeren van een eigen authenticatie- en autorisatiesysteem in hun applicaties?
  \item Welke oplossingen kunnen worden geïdentificeerd om deze moeilijkheden te verhelpen?
  \item Hoe kunnen deze oplossingen worden geïmplementeerd in een functioneel systeem?
\end{itemize}
Hieruit volgt de onderzoeksvraag: "Authenticatie en Autorisatie in applicatieontwikkeling, hoe kan dit makkelijk worden aangeboden: Een diepgaande analyse en implementatie".
\newline
\newline
De onderzoeksdoelstelling is het identificeren van praktische oplossingen en het implementeren van een authenticatie- en autorisatiesysteem 
dat gemakkelijk toegankelijk is voor ontwikkelaars. Het concrete eindresultaat van dit onderzoek is een functioneel systeem met een API-backend 
en een dashboard-frontend, evenals een rapport met aanbevelingen en best practices voor ontwikkelaars.
\newline
\newline
Door deze bachelorproef als succesvol te beschouwen, zou het ontwikkelde systeem niet alleen moeten voldoen aan de verwachtingen van gebruikers, 
maar ook aantoonbaar bijdragen aan het vereenvoudigen en optimaliseren van het proces van authenticatie en autorisatie in applicatieontwikkeling.

%---------- Stand van zaken ---------------------------------------------------

\section{Stand van Zaken}%
\label{sec:state-of-the-art}

Het huidige onderzoeksdomein van authenticatie en autorisatie in applicatieontwikkeling is van cruciaal belang, aangezien het de basis vormt voor het
waarborgen van de beveiliging en privacy van gebruikers. In deze sectie wordt de state of the art rondom dit onderwerp onderzocht, met een focus op
relevante literatuur en bestaande oplossingen.

\subsection{Authenticatie en Autorisatie in Applicatieontwikkeling}

Authenticatie en autorisatie zijn twee essentiële stappen in het toegangsbeheer proces van een applicatie. Authenticatie verifieert de identiteit van
een gebruiker, terwijl autorisatie bepaalt welke acties een geauthenticeerd gebruiker mag uitvoeren. De huidige authenticatie methoden variëren van
traditionele wachtwoordgebaseerde systemen tot geavanceerde biometrische authenticatie, multi-factor authenticatie (MFA), certificaatgebaseerde authenticatie, OAuth 2.0, OpenID Connect en meer.
Autorisatie methoden omvatten rollen-gebaseerde toegangscontrole, attribuut-gebaseerde toegangscontrole, beleids-gebaseerde toegangscontrole, OAuth 2.0 frameworks, JSON Web Tokens (JWT)
en meer~\autocite{Hardt2012}.

\subsection{Huidige Uitdagingen}

Ondanks de beschikbaarheid van verschillende authenticatie- en autorisatiemethoden, blijven er uitdagingen bestaan. Gebruikerservaring, beheer van
gebruikersidentiteiten, en de implementatie van sterke beveiligingsmaatregelen zijn enkele van de voornaamste aandachtspunten. Bovendien kunnen de
complexiteit en de diversiteit van applicaties het proces voor ontwikkelaars bemoeilijken~\autocite{Bakar2013}.

\subsection{Bestaande Oplossingen}

Diverse oplossingen en frameworks zijn ontwikkeld om ontwikkelaars te ondersteunen bij het implementeren van veilige authenticatie- en autorisatiesystemen.
Populaire frameworks zoals Firebase Authentication, Auth0, Keycloak, Amazon Cognito en Okta bieden diensten aan die specifiek gericht zijn op het vereenvoudigen van deze
processen. Deze oplossingen variëren in functies, integraties, en ondersteunde authenticatiemethoden.

\subsection{Toekomstige Ontwikkelingen}

Met de voortdurende evolutie van applicatieontwikkeling worden nieuwe uitdagingen en kansen geïdentificeerd. Toekomstige ontwikkelingen kunnen zich
richten op verbeterde methoden voor biometrische authenticatie, naadloze integratie van verschillende authenticatiemiddelen, en de toepassing van
artificial intelligence voor het detecteren van ongebruikelijke activiteiten en het voorkomen van ongeautoriseerde toegang~\autocite{Annadurai2022}.

\subsection{Open Vragen en Onderzoeksdomeinen}

Ondanks de vooruitgang in authenticatie- en autorisatietechnologieën blijven er open vragen en onderzoeksdomeinen bestaan. Hoe kunnen ontwikkelaars
eenvoudigere, maar toch veilige, systemen implementeren? Welke methoden bieden een optimale balans tussen beveiliging en gebruikersgemak? Hoe kan
het proces van authenticatie en autorisatie geoptimaliseerd worden voor verschillende soorten applicaties?

Deze vragen dienen als basis voor de verdere uitwerking van dit onderzoek en zullen bijdragen aan het identificeren van praktische oplossingen en
best practices voor ontwikkelaars.

\section{Methodologie}%
\label{sec:methodologie}

De methodologie voor dit onderzoek omvat verschillende stappen, variërend van literatuurstudie tot de ontwikkeling en evaluatie van een 
authenticatie- en autorisatiesysteem. Hieronder worden de belangrijkste methoden en stappen beschreven:

\subsection{Literatuurstudie}

De literatuurstudie vormt het fundament van dit onderzoek. Hierbij wordt diepgaand onderzoek gedaan naar de actuele stand van zaken op het gebied 
van authenticatie en autorisatie in applicatieontwikkeling. De literatuurstudie richt zich specifiek op bestaande methoden, protocollen, frameworks, 
en technologieën die binnen vergelijkbare contexten van authenticatiesystemen worden toegepast. Concrete voorbeelden van dergelijke methoden en 
technologieën omvatten OAuth 2.0, OpenID Connect, JWT (JSON Web Tokens), OTP (One Time Passwords), SAML (Security Assertion Markup Language), en 
biometrische authenticatie. Ook zullen bestaande authenticatie- en autorisatie aanbieders, zoals Firebase Authentication, Auth0, Okta en Amazon Cognito
worden onderzocht. Het doel hiervan is om inzicht te krijgen in de huidige uitdagingen en oplossingen binnen het onderzoeksdomein.
Uiteindelijk zal de literatuurstudie zich richten op het selecteren van een geschikte combinatie van methoden en technologieën voor de ontwikkeling
van een eigen authenticatie- en autorisatiesysteem. Ook zullen een aantal bestaande aanbieders worden geselecteerd als referentiekader voor de
ontwikkeling van de Proof-of-Concept (PoC). Hierbij zal altijd duidelijk worden aangegeven waarom bepaalde keuzes zijn gemaakt.
Door relevante wetenschappelijke artikelen, boeken en onlinebronnen te raadplegen, beoogt deze literatuurstudie een volledig begrip te verkrijgen 
van de specifieke uitdagingen en oplossingen binnen het domein van authenticatie en autorisatie. Hierbij zal extra aandacht worden besteed aan de 
keuze voor een eigen authenticatiemechanisme, waarbij de bestaande oplossingen als referentiekader dienen.

\subsection{Requirements analyse}

Om de behoeften van de doelgroep in kaart te brengen, zal een requirements analyse worden uitgevoerd.
De analyse zal zich richten op de vereisten voor een authenticatie- en 
autorisatiesysteem, met specifieke aandacht voor gebruiksvriendelijkheid, beveiliging, en integratiemogelijkheden.

\subsection{Long list} 

In de long list fase worden de bestaande authenticatie- en autorisatie frameworks geëvalueerd op basis van de literatuurstudie. Hieruit volgt een lijst van
authenticatie- en autorisatie frameworks die voldoen aan de vereisten van de requirements analyse.
\newline
Als resultaat van deze fase hebben we een duidelijke lijst met de interessantste alternatieven.

\subsection{Short list}

In de short list fase worden de bestaande authenticatie- en autorisatie frameworks verder geëvalueerd op basis van de long list. Hieruit volgt een lijst van
authenticatie- en autorisatie frameworks die het meest geschikt zijn voor de ontwikkeling van de Proof-of-Concept (PoC).
\newline
Als resultaat van deze fase hebben we een duidelijke lijst met de meest geschikte alternatieven.

\subsection{Proof-of-Concept (PoC) Ontwikkeling}

Op basis van de bevindingen uit de literatuurstudie en requirements analyse zal een Proof-of-Concept (PoC) worden ontwikkeld. Hierbij zal de focus 
liggen op het onderzoeken van de implementatie van een authenticatie- en autorisatiesysteem, met specifieke aandacht voor het aspect van 
het zelf opzetten van een eigen authenticatiesysteem.
De technologische keuzes voor de ontwikkeling van de PoC zullen niet vooraf worden vastgesteld, maar zullen het resultaat zijn van het onderzoek 
naar de eenvoudigheid van het opzetten van een op maat gemaakt authenticatiesysteem. De implementatie kan gebruikmaken van verschillende frameworks 
en technologieën, en tijdens de ontwikkelingsfase zal aandacht worden besteed aan het volgen van best practices en beveiligingsrichtlijnen.

\subsection{Implementatie Evaluatie}

De ontwikkelde Proof-of-Concept (PoC) zal grondig worden geëvalueerd in een realistische omgeving. Deze evaluatie omvat het testen van het systeem in 
diverse scenario's, waaronder verschillende types applicaties zoals desktop-, web- en mobiele apps. De evaluatie zal specifiek gericht zijn op meetbare 
criteria zoals prestaties, gebruiksvriendelijkheid en beveiliging. Concreet worden parameters zoals responstijd, laadtijden, en gebruiksgemak gemeten en 
geanalyseerd. Daarnaast zullen beveiligingsaspecten, zoals de robuustheid tegen mogelijke aanvallen door middel van penetratietesten, ook worden beoordeeld. 
De resultaten van deze evaluatie zullen dienen als basis voor het identificeren van mogelijke verbeteringen aan het systeem.

\subsection{Rapportage en Aanbevelingen}

Het onderzoek zal worden afgesloten met de rapportage van de bevindingen en aanbevelingen. Het rapport zal een gedetailleerde beschrijving bevatten 
van het ontwikkelde systeem, de evaluatieresultaten, en praktische aanbevelingen voor ontwikkelaars. Het doel is om een waardevolle bijdrage te leveren 
aan de bestaande kennis in het vakgebied en concrete richtlijnen te bieden voor de implementatie van authenticatie- en autorisatiesystemen.

\subsection{Tijdschatting}

\begin{itemize}
\item Literatuurstudie: 4 weken
\item Requirements analyse: 2 weken
\item PoC Ontwikkeling: 8 weken
\item Implementatie Evaluatie: 4 weken
\item Rapportage en Aanbevelingen: 6 weken
\end{itemize}

Deze tijdschatting is indicatief en kan tijdens het onderzoek worden aangepast op basis van de voortgang en eventuele onverwachte uitdagingen.

\section{Verwachte Resultaten, Conclusie}%
\label{sec:verwachte_resultaten}

De verwachte resultaten van dit onderzoek omvatten:

\begin{enumerate}
    \item \textbf{Een Authenticatie- en Autorisatiesysteem:} 
    De ontwikkeling van een Proof-of-Concept (PoC) dat aantoont hoe een authenticatie- en autorisatiesysteem kan worden geïmplementeerd met 
    behulp van een ontwikkelde front- en backend.

    \item \textbf{Optimalisatie van gebruikerservaring:} 
    Verwacht wordt dat het geïmplementeerde systeem zal bijdragen aan een optimale gebruikerservaring door middel van intuïtieve interfaces en 
    vlotte authenticatieprocessen.

    \item \textbf{Robuuste beveiliging:} 
    Het verwachte resultaat omvat een geoptimaliseerde beveiliging door implementatie van best practices en richtlijnen voor authenticatie- en 
    autorisatieprocessen.

    \item \textbf{Praktische aanbevelingen:}
    Het onderzoeksrapport zal concrete aanbevelingen bevatten voor ontwikkelaars over hoe ze authenticatie- en autorisatiesystemen kunnen 
    implementeren in diverse applicaties.
\end{enumerate}

De doelgroep van dit onderzoek, bestaande uit IT professionals en ontwikkelaars, zal profiteren van de verwachte resultaten door toegang te krijgen 
tot een praktisch implementeerbaar systeem en waardevolle richtlijnen. De meerwaarde van deze bachelorproef ligt in het bieden van tastbare 
oplossingen voor de uitdagingen rondom authenticatie en autorisatie, waardoor ontwikkelaars in staat worden gesteld veilige en gebruiksvriendelijke 
applicaties te ontwerpen.
