%---------- Inleiding ---------------------------------------------------------


\section{Introductie}%
\label{sec:introductie}

Deze bachelorproef richt zich op het verbeteren van de authenticatie en autorisatie in applicatieontwikkeling, met een specifieke focus op het 
vereenvoudigen van deze processen voor developers. Het startpunt van deze bachelorproef komt voort uit mijn ervaring als IT'er en ontwikkelaar, 
waar ik aanvankelijk moeite ondervond met het beveiligen en beschikbaar stellen van applicaties voor gebruikers zonder ongeautoriseerde toegang.
\newline
\newline
De doelgroep van deze bachelorproef bestaat uit IT professionals en ontwikkelaars die betrokken zijn bij het ontwerp en de ontwikkeling van
applicaties. De doelgroep heeft een basiskennis van applicatieontwikkeling en beveiliging, maar heeft mogelijk geen ervaring met het ontwikkelen
van een eigen authenticatie- en autorisatiesysteem en heeft dus behoefte aan een oplossing die gemakkelijk te implementeren is in hun applicaties.
\newline
\newline
De centrale probleemstelling van dit onderzoek is het identificeren van de moeilijkheden die ontwikkelaars ondervinden bij het implementeren van
een eigen authenticatie- en autorisatiesysteem in hun applicaties. De onderzoeksvragen die hieruit voortvloeien zijn:
\begin{itemize}
  \item Wat zijn de moeilijkheden die ontwikkelaars ondervinden bij het implementeren van een eigen authenticatie- en autorisatiesysteem in hun applicaties?
  \item Welke oplossingen kunnen worden geïdentificeerd om deze moeilijkheden te verhelpen?
  \item Hoe kunnen deze oplossingen worden geïmplementeerd in een functioneel systeem?
\end{itemize}
Hieruit volgt de onderzoeksvraag: "De moeilijkheid van Authenticatie en Autorisatie in applicatieontwikkeling, hoe kunnen we dit makkelijk aanbieden: 
Een diepgaande analyse en implementatie".
\newline
\newline
De onderzoeksdoelstelling is het identificeren van praktische oplossingen en het implementeren van een verbeterd authenticatie- en autorisatiesysteem 
dat gemakkelijk toegankelijk is voor ontwikkelaars. Het concrete eindresultaat van dit onderzoek is een functioneel systeem met een Node.js API-backend 
en een React.js-frontend, evenals een rapport met aanbevelingen en best practices voor ontwikkelaars.
\newline
\newline
Door deze bachelorproef als succesvol te beschouwen, zou het ontwikkelde systeem niet alleen moeten voldoen aan de verwachtingen van gebruikers, 
maar ook aantoonbaar bijdragen aan het vereenvoudigen en optimaliseren van het proces van authenticatie en autorisatie in applicatieontwikkeling.

%---------- Stand van zaken ---------------------------------------------------

\section{Stand van Zaken}%
\label{sec:state-of-the-art}

Het huidige onderzoeksdomein van authenticatie en autorisatie in applicatieontwikkeling is van cruciaal belang, aangezien het de basis vormt voor het
waarborgen van de beveiliging en privacy van gebruikers. In deze sectie wordt de state of the art rondom dit onderwerp onderzocht, met een focus op
relevante literatuur en bestaande oplossingen.

\subsection{Authenticatie en Autorisatie in Applicatieontwikkeling}

Authenticatie en autorisatie zijn twee essentiële stappen in het toegangsbeheer proces van een applicatie. Authenticatie verifieert de identiteit van
een gebruiker, terwijl autorisatie bepaalt welke acties een geauthenticeerd gebruiker mag uitvoeren. Verschillende methoden en protocollen worden
gebruikt voor deze processen, waaronder gebruikersnaam/wachtwoord, OAuth 2.0 en OpenID Connect~\autocite{Hardt2012}.

\subsection{Huidige Uitdagingen}

Ondanks de beschikbaarheid van verschillende authenticatie- en autorisatiemethoden, blijven er uitdagingen bestaan. Gebruikerservaring, beheer van
gebruikersidentiteiten, en de implementatie van sterke beveiligingsmaatregelen zijn enkele van de voornaamste aandachtspunten. Bovendien kunnen de
complexiteit en de diversiteit van applicaties het proces voor ontwikkelaars bemoeilijken~\autocite{Bakar2013}.

\subsection{Bestaande Oplossingen}

Diverse oplossingen en frameworks zijn ontwikkeld om ontwikkelaars te ondersteunen bij het implementeren van veilige authenticatie- en autorisatiesystemen.
Populaire frameworks zoals Firebase Authentication, Auth0, en Keycloak bieden diensten aan die specifiek gericht zijn op het vereenvoudigen van deze
processen. Deze oplossingen variëren in functies, integraties, en ondersteunde authenticatiemethoden.

\subsection{Toekomstige Ontwikkelingen}

Met de voortdurende evolutie van applicatieontwikkeling worden nieuwe uitdagingen en kansen geïdentificeerd. Toekomstige ontwikkelingen kunnen zich
richten op verbeterde methoden voor biometrische authenticatie, naadloze integratie van verschillende authenticatiemiddelen, en de toepassing van
artificial intelligence voor het detecteren van ongebruikelijke activiteiten en het voorkomen van ongeautoriseerde toegang~\autocite{Annadurai2022}.

\subsection{Open Vragen en Onderzoeksdomeinen}

Ondanks de vooruitgang in authenticatie- en autorisatietechnologieën blijven er open vragen en onderzoeksdomeinen bestaan. Hoe kunnen ontwikkelaars
eenvoudigere, maar toch veilige, systemen implementeren? Welke methoden bieden een optimale balans tussen beveiliging en gebruikersgemak? Hoe kan
het proces van authenticatie en autorisatie geoptimaliseerd worden voor verschillende soorten applicaties?

Deze vragen dienen als basis voor de verdere uitwerking van dit onderzoek en zullen bijdragen aan het identificeren van praktische oplossingen en
best practices voor ontwikkelaars.

\section{Methodologie}%
\label{sec:methodologie}

De methodologie voor dit onderzoek omvat verschillende stappen, variërend van literatuurstudie tot de ontwikkeling en evaluatie van een verbeterd 
authenticatie- en autorisatiesysteem. Hieronder worden de belangrijkste methoden en stappen beschreven:

\subsection{Literatuurstudie}

Een grondige literatuurstudie zal de basis vormen van dit onderzoek. Hierbij wordt uitgebreid onderzocht wat de huidige stand van zaken is op het 
gebied van authenticatie en autorisatie in applicatieontwikkeling. De literatuurstudie zal zich richten op bestaande methoden, protocollen, frameworks, 
en technologieën die worden gebruikt in vergelijkbare contexten. Relevante wetenschappelijke artikelen, boeken en online bronnen zullen worden 
geraadpleegd om een volledig begrip te krijgen van de uitdagingen en oplossingen in dit domein.

\subsection{Requirements analyse}

Om de behoeften van de doelgroep in kaart te brengen, zal een requirements analyse worden uitgevoerd. Dit omvat mogelijk interviews met ontwikkelaars 
en IT professionals die betrokken zijn bij applicatieontwikkeling. De analyse zal zich richten op de vereisten voor een verbeterd authenticatie- en 
autorisatiesysteem, met specifieke aandacht voor gebruiksvriendelijkheid, beveiliging, en integratiemogelijkheden.

\subsection{Proof-of-Concept (PoC) Ontwikkeling}

Op basis van de bevindingen uit de literatuurstudie en requirements analyse zal een Proof-of-Concept (PoC) worden ontwikkeld. Dit omvat de implementatie 
van een verbeterd authenticatie- en autorisatiesysteem met behulp van Node.js voor de backend en React.js voor de frontend. Tijdens de ontwikkeling zal 
rekening worden gehouden met best practices en beveiligingsrichtlijnen.

\subsection{Implementatie Evaluatie}

De ontwikkelde PoC zal worden geëvalueerd in een realistische omgeving. Dit omvat het testen van het systeem in verschillende scenario's, waaronder 
verschillende soorten applicaties (desktop, web, mobiel). De evaluatie zal zich richten op prestaties, gebruiksvriendelijkheid en beveiliging. 
Mogelijke verbeteringen zullen worden geïdentificeerd.

\subsection{Rapportage en Aanbevelingen}

Het onderzoek zal worden afgesloten met de rapportage van de bevindingen en aanbevelingen. Het rapport zal een gedetailleerde beschrijving bevatten 
van het ontwikkelde systeem, de evaluatieresultaten, en praktische aanbevelingen voor ontwikkelaars. Het doel is om een waardevolle bijdrage te leveren 
aan de bestaande kennis in het vakgebied en concrete richtlijnen te bieden voor de implementatie van verbeterde authenticatie- en autorisatiesystemen.

\subsection{Tijdschatting}

\begin{itemize}
\item Literatuurstudie: 4 weken
\item Requirements analyse: 2 weken
\item PoC Ontwikkeling: 8 weken
\item Implementatie Evaluatie: 4 weken
\item Rapportage en Aanbevelingen: 6 weken
\end{itemize}

Deze tijdschatting is indicatief en kan tijdens het onderzoek worden aangepast op basis van de voortgang en eventuele onverwachte uitdagingen.

\section{Verwachte Resultaten, Conclusie}%
\label{sec:verwachte_resultaten}

De verwachte resultaten van dit onderzoek omvatten:

\begin{enumerate}
    \item \textbf{Een verbeterd Authenticatie- en Autorisatiesysteem:} 
    De ontwikkeling van een Proof-of-Concept (PoC) dat aantoont hoe een verbeterd authenticatie- en autorisatiesysteem kan worden geïmplementeerd met 
    behulp van Node.js voor de backend en React.js voor de frontend.

    \item \textbf{Optimalisatie van gebruikerservaring:} 
    Verwacht wordt dat het geïmplementeerde systeem zal bijdragen aan een verbeterde gebruikerservaring door middel van intuïtieve interfaces en 
    vlotte authenticatieprocessen.

    \item \textbf{Verbeterde beveiliging:} 
    Het verwachte resultaat omvat een verbeterde beveiliging door implementatie van best practices en richtlijnen voor authenticatie- en 
    autorisatieprocessen.

    \item \textbf{Praktische aanbevelingen:}
    Het onderzoeksrapport zal concrete aanbevelingen bevatten voor ontwikkelaars over hoe ze verbeterde authenticatie- en autorisatiesystemen kunnen 
    implementeren in diverse applicaties.
\end{enumerate}

De doelgroep van dit onderzoek, bestaande uit IT professionals en ontwikkelaars, zal profiteren van de verwachte resultaten door toegang te krijgen 
tot een praktisch implementeerbaar systeem en waardevolle richtlijnen. De meerwaarde van deze bachelorproef ligt in het bieden van tastbare 
oplossingen voor de uitdagingen rondom authenticatie en autorisatie, waardoor ontwikkelaars in staat worden gesteld veilige en gebruiksvriendelijke 
applicaties te ontwerpen.
