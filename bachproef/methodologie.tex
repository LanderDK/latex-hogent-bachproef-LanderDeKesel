%%=============================================================================
%% Methodologie
%%=============================================================================

\chapter{\IfLanguageName{dutch}{Methodologie}{Methodology}}%
\label{ch:methodologie}

%% TODO: In dit hoofstuk geef je een korte toelichting over hoe je te werk bent
%% gegaan. Verdeel je onderzoek in grote fasen, en licht in elke fase toe wat
%% de doelstelling was, welke deliverables daar uit gekomen zijn, en welke
%% onderzoeksmethoden je daarbij toegepast hebt. Verantwoord waarom je
%% op deze manier te werk gegaan bent.
%% 
%% Voorbeelden van zulke fasen zijn: literatuurstudie, opstellen van een
%% requirements-analyse, opstellen long-list (bij vergelijkende studie),
%% selectie van geschikte tools (bij vergelijkende studie, "short-list"),
%% opzetten testopstelling/PoC, uitvoeren testen en verzamelen
%% van resultaten, analyse van resultaten, ...
%%
%% !!!!! LET OP !!!!!
%%
%% Het is uitdrukkelijk NIET de bedoeling dat je het grootste deel van de corpus
%% van je bachelorproef in dit hoofstuk verwerkt! Dit hoofdstuk is eerder een
%% kort overzicht van je plan van aanpak.
%%
%% Maak voor elke fase (behalve het literatuuronderzoek) een NIEUW HOOFDSTUK aan
%% en geef het een gepaste titel.


\section{Vergelijkingscriteria OAuth 2.0 Docker Image's}%
\label{subsec:vergelijkingscriteria}

\begin{enumerate}
  \item Gebruiksgemak en configuratie: Hoe gemakkelijk is het om de Docker-image in te stellen en te configureren voor je specifieke gebruiksscenario? Worden er duidelijke instructies geleverd?

  \item Ondersteunde OAuth 2.0-functies: Controleer welke functies van het OAuth 2.0 protocol worden ondersteund door de verschillende implementaties. Dit kan onder meer autorisatiecodes, impliciete access tokens, client credentials en refresh tokens omvatten.
  
  \item Ondersteuning voor andere protocollen: Sommige auth server implementaties ondersteunen naast OAuth 2.0 ook andere protocollen zoals OpenID Connect. Het kan handig zijn om te beoordelen welke protocollen worden ondersteund als je behoeften hebt die verder gaan dan alleen OAuth 2.0.
  
  \item Schaalbaarheid en prestaties: Hoe schaalbaar is de auth server? Kan het omgaan met een groot aantal gelijktijdige gebruikers en verzoeken? Zijn er prestatie benchmarks beschikbaar?
  
  \item Aanpasbaarheid en uitbreidbaarheid: Kun je de functionaliteit van de auth server aanpassen of uitbreiden met behulp van plugins of aangepaste code? Hoe gemakkelijk is het om deze aanpassingen te maken?
  
  \item Documentatie en community ondersteuning: Is er uitgebreide documentatie beschikbaar voor de Docker-image en de bijbehorende auth server? Is er een actieve community waar je vragen kunt stellen en ondersteuning kunt krijgen?
  
  \item Beveiligingsfuncties: Welke beveiligingsfuncties biedt de auth server? Wordt er bijvoorbeeld ondersteuning geboden voor multifactor authenticatie, JWT verificatie of integratie met externe identiteitsproviders?
  
  \item Onderhoud en updates: Wordt de Docker-image regelmatig bijgewerkt met bug fixes en beveiliging patches? Hoe actief is de ontwikkeling van de auth server?
  
  \item Beschikbaarheid van integraties: Controleer of de auth server integraties biedt met populaire frameworks, bibliotheken en platforms die je gebruikt in je applicatie stack.
  
  \item Kosten en licentie: Sommige auth server implementaties zijn gratis en open source, terwijl andere een commercieel licentiemodel hebben. Overweeg welke kosten er zijn verbonden aan het gebruik van de verschillende implementaties en of de licentievoorwaarden aansluiten bij je behoeften.
\end{enumerate}


\section{Long list}%
\label{subsec:long-list}
\begin{table}[htbp]
  \centering
  \caption{OAuth 2.0-authenticatieservers en Docker-image beschikbaarheid}
  \label{tab:oauth_servers}
  \begin{adjustbox}{width=1\textwidth}
  \begin{tabular}{@{}llll@{}}
    \toprule
    Naam          & Website                               & Beschrijving                                                                   & Docker-image beschikbaar \\ \midrule
    Keycloak      & \texttt{https://www.keycloak.org/}     & Open source identiteits- en toegangsbeheer voor moderne applicaties en services. & Ja                        \\
    Hydra         & \texttt{https://www.ory.sh/hydra/}     & OAuth 2.0 en OpenID Connect-server met krachtige functies voor authenticatie en autorisatie. & Ja                        \\
    Oathkeeper & \texttt{https://www.ory.sh/oathkeeper/} & Identity \& Access Proxy (IAP) gebouwd op top van Ory Hydra en Ory Keto. & Ja                        \\
    Gluu          & \texttt{https://www.gluu.org/}         & Open source IAM-platform voor web- en mobiele applicaties.                   & Ja (community images)    \\
    Apereo CAS    & \texttt{https://apereo.github.io/cas/} & Central Authentication Service (CAS) voor authenticatie en autorisatie.      & Ja                        \\
    Dex           & \texttt{https://dexidp.io/}            & Open source OIDC-provider met LDAP-ondersteuning.                             & Ja                        \\
    FusionAuth    & \texttt{https://fusionauth.io/}        & Identity and access management voor developers.                               & Ja                        \\
    LemonLDAP::NG & \texttt{https://lemonldap-ng.org/}     & Open source toegangsbeheer voor webapplicaties.                               & Ja                        \\
    Keycloak Gatekeeper & \texttt{https://www.keycloak.org/docs/latest/securing\_apps/} & Een authenticatie-gateway die werkt met Keycloak.                     & Ja                        \\
    IdentityServer & \texttt{https://identityserver.io/}    & OpenID Connect- en OAuth 2.0-protocolserver voor ASP.NET Core.               & Nee                       \\
    Apache Oltu   & \texttt{https://oltu.apache.org/}      & OAuth 2.0-bibliotheken voor Java.                                             & Nee                       \\
    UAA           & \texttt{https://github.com/cloudfoundry/uaa} & Open source identiteitsbeheerservice voor Cloud Foundry.                & Nee                       \\
    Okta          & \texttt{https://www.okta.com/}         & Identity Cloud-service met ondersteuning voor OAuth 2.0 en OpenID Connect.    & Nee                       \\
    Auth0         & \texttt{https://auth0.com/}            & Identity-platform voor ontwikkelaars met OAuth 2.0 en OpenID Connect-ondersteuning. & Nee                       \\
    AWS Cognito   & \texttt{https://aws.amazon.com/cognito/} & Identity-service van Amazon Web Services met ondersteuning voor OAuth 2.0 en OpenID Connect. & Nee                       \\ \bottomrule
  \end{tabular}
  \end{adjustbox}
\end{table}
\begin{table}[htbp]
  \centering
  \caption{Alternatieven beoordelen op basis van vergelijkingscriteria}
  \label{tab:oauth_servers}
  \begin{adjustbox}{width=1\textwidth}
  \begin{tabular}{@{}lllllllllllll@{}}
    \toprule
    Naam          & Gebruiksgemak & OAuth 2.0 & Andere protocollen & Schaalbaarheid & Aanpasbaarheid & Documentatie & Beveiliging & Onderhoud & Integraties & Kosten & Totaal \\ 
                   &                & functies &                     & en prestaties   & en uitbreidbaarheid & en community-ondersteuning & functies & en updates & beschikbaarheid & en licentie &        \\
    \midrule
    Keycloak      & Ja             & Ja        & Ja                  & Nee             & Ja               & Ja             & Ja         & Ja         & Ja             & Ja          & 9      \\
    Hydra         & Ja             & Ja        & Ja                  & Ja              & Ja               & Ja             & Ja         & Ja         & Ja             & Ja          & 10      \\
    Oathkeeper & Ja            & Ja        & Ja                  & Ja              & Ja               & Ja             & Ja         & Ja         & Ja             & Ja          & 10      \\
    Gluu          & Ja             & Ja        & ?                   & ?               & Ja               & Ja             & Ja         & Ja         & Ja             & Ja          & 8      \\
    Apereo CAS    & Ja             & Ja        & ?                   & ?               & Ja               & Ja             & Ja         & Ja         & Ja             & Ja          & 8      \\
    Dex           & Ja             & Ja        & ?                   & ?               & Ja               & Ja             & Ja         & Ja         & Ja             & Ja          & 8      \\
    FusionAuth    & Ja             & Ja        & ?                   & ?               & Ja               & Ja             & Ja         & Ja         & Ja             & Ja          & 8      \\
    LemonLDAP::NG & Ja             & Ja        & ?                   & ?               & Ja               & Ja             & Ja         & Ja         & Ja             & Ja          & 8      \\
    Keycloak Gatekeeper & Ja       & Ja        & ?                   & ?               & Ja               & Ja             & Ja         & Ja         & Ja             & Ja          & 8      \\
    IdentityServer & ?             & ?         & ?                   & ?               & ?                & ?              & ?          & ?          & ?              & ?           & ?      \\
    Apache Oltu   & ?             & ?         & ?                   & ?               & ?                & ?              & ?          & ?          & ?              & ?           & ?      \\
    UAA           & ?             & ?         & ?                   & ?               & ?                & ?              & ?          & ?          & ?              & ?           & ?      \\
    Okta          & ?             & ?         & ?                   & ?               & ?                & ?              & ?          & ?          & ?              & ?           & ?      \\
    Auth0         & ?             & ?         & ?                   & ?               & ?                & ?              & ?          & ?          & ?              & ?           & ?      \\
    AWS Cognito   & ?             & ?         & ?                   & ?               & ?                & ?              & ?          & ?          & ?              & ?           & ?      \\
    \bottomrule
  \end{tabular}
  \end{adjustbox}
\end{table}


\section{Short list}%
\label{subsec:short-list}
Hier is een shortlist van de meest veelbelovende alternatieven:
\begin{enumerate}
    \item Keycloak
    \item Hydra
    \item Oathkeeper
\end{enumerate}

Deze drie alternatieven hebben het hoogste totaal aantal voldane vereisten en hebben ook Docker-images beschikbaar.

Voor de proof-of-concept kan worden voorgesteld om één van deze drie alternatieven te kiezen en een eenvoudige implementatie op te zetten om een basaal scenario te testen. Dit kan bijvoorbeeld inhouden:

\begin{itemize}
    \item Het opzetten van een OAuth 2.0-authenticatieserver met het gekozen alternatief.
    \item Het implementeren van een eenvoudige client-applicatie die de authenticatie via OAuth 2.0 integreert.
    \item Het uitvoeren van enkele basisauthenticatie- en autorisatietests om te controleren of het alternatief voldoet aan de vereisten zoals gebruikersbeheer, toegangscontrole en veiligheid.
\end{itemize}

Hierdoor kan de werking van het gekozen alternatief beter begrepen worden en bepalen of het geschikt is voor verdere implementatie in je project.


\section{Short list alternatieven testen}
\label{subsec:short-list-alternatieven-testen}
In deze sectie wordt er verder verdiept in de drie alternatieven, namelijk Keycloak, Hydra en Oathkeeper.
Het doel is de alternatieven van de short list te testen en te evalueren op basis van de eerder vastgestelde criteria. Indien er pijnpunten of tekortkomingen worden vastgesteld, zullen deze worden opgesomd en zullen er suggesties worden gegeven voor mogelijke verbeteringen.

\subsection{Keycloak}%
\label{subsubsec:keycloak}
Als eerste werd er genavigeerd naar de documentatie, hier werd snel de download gevonden voor de Docker container. Na het downloaden van de container werd deze gestart en werd de webinterface geopend. De interface was zeer gebruiksvriendelijk en intuïtief, met duidelijke instructies voor het instellen van gebruikers, clients en scopes. Het was ook mogelijk om plugins en themas toe te voegen om de functionaliteit en het uiterlijk van de auth server aan te passen. De documentatie was uitgebreid en er was een actieve community die vragen kon beantwoorden en ondersteuning kon bieden. Keycloak ondersteunde een breed scala aan OAuth 2.0-functies, waaronder autorisatiecodes, impliciete access tokens, client credentials en refresh tokens. Het bood ook ondersteuning voor andere protocollen zoals OpenID Connect en SAML. De beveiligingsfuncties waren robuust, met ondersteuning voor multifactor authenticatie en integratie met externe identiteitsproviders. Keycloak werd regelmatig bijgewerkt en onderhouden en had integraties met populaire frameworks en platforms. Het was gratis en open source, wat het een aantrekkelijke optie maakte voor ontwikkelaars die op zoek waren naar een krachtige en aanpasbare auth server.
Het was zelfs niet nodig om een proof-of-concept op te zetten, omdat u de SPA-testapplicatie op de Keycloak-website kunt gebruiken. Dit is een geweldige manier om de functionaliteit van Keycloak te verkennen en te begrijpen hoe het werkt in de praktijk en duurt slechts enkele minuten om op te zetten.
Met de ervaring die is opgedaan met Keycloak, kan worden geconcludeerd dat deze oplossing zeer gebruiksvriendelijk is met veel configuratie mogelijkheden en een zeer uitgebreide documentatie heeft. De ``Getting started with Docker'' handleiding is zeer duidelijk en eenvoudig te volgen. De webinterface is intuïtief en biedt een breed scala aan functies voor het beheren van gebruikers, clients en scopes. De ondersteuning voor OAuth 2.0 en andere protocollen is uitgebreid en de beveiligingsfuncties zijn robuust. De actieve community en regelmatige updates maken Keycloak een aantrekkelijke optie voor ontwikkelaars die op zoek zijn naar een krachtige en aanpasbare auth server.

\subsection{Hydra}%
\label{subsubsec:hydra}
Opnieuw werd er eerst gezocht naar documentatie, dit werd gevonden onder de ``Self-hosting'' sectie. De download werd hier ook gevonden voor de Docker container.
Maar het opzetten en zelfs starten van de container was niet eenvoudig in vergelijking met Keycloak. Na dezelfde hoeveelheid tijd te hebben besteed aan Hydra als aan Keycloak, was het nog steeds niet gelukt om de webinterface te openen en de auth server te configureren. 
Dit was een groot pijnpunt, omdat het niet mogelijk was om de functionaliteit van Hydra te verkennen en te testen. Dit maakte het al snel minder aantrekkelijk om Hydra te gebruiken boven Keycloak. Het was duidelijk dat Hydra een steilere leercurve had en meer configuratie vereiste om aan de slag te gaan. Dit kan een obstakel vormen voor ontwikkelaars die op zoek zijn naar een snelle en eenvoudige manier om een auth server op te zetten.
De documentatie lijk wel uitgebreid, maar er zijn veel meer stappen en configuratie vereist om Hydra aan de praat te krijgen. 
Echter zijn er wel een aantal methoden om in contact te komen met de community, dus het is mogelijk dat er hulp kan worden gevonden om Hydra aan de praat te krijgen.

\subsection{Oathkeeper}%
\label{subsubsec:oathkeeper}
De documentatie was beter en de stappen waren gedetailleerder en het was mogelijk een aantal dingen uit te testen en een vaststelling te maken: namelijk dat dit enkel gericht is op API-gateway functionaliteit. 
Dit wil zeggen dat er geen UI is omdat er vooral moet worden geconfigureerd via de command line. Dit kan een obstakel vormen voor ontwikkelaars die op zoek zijn naar een gebruiksvriendelijke en intuïtieve manier om een auth server op te zetten.
Ondanks is dit wel een goede oplossing, maar valt het een beetje buiten de scope van dit onderzoek, omdat er geen UI is en het vooral is op API-gateway functionaliteit.


\section{Docker Images vergelijken}%
\label{subsec:docker-images-vergelijken}
Op basis van de eerder genoemde criteria, werden de volgende Docker Images vergeleken: Keycloak, Hydra en Oathkeeper.

\begin{enumerate}
  \item Gebruiksgemak en configuratie:
  \begin{itemize}
    \item Keycloak wordt over het algemeen beschouwd als gebruiksvriendelijk met een intuïtieve gebruikersinterface voor configuratie.
    \item Hydra vereist wat meer configuratie, vooral als het gaat om het opzetten van complexe autorisatie flows.
    \item Oathkeeper kan uitdagender zijn om te configureren vanwege de focus op API-gateway functionaliteit, hoewel het krachtige mogelijkheden biedt.
  \end{itemize}
  
  \item Ondersteunde OAuth 2.0-functies:
  \begin{itemize}
    \item Keycloak biedt een breed scala aan functies, waaronder autorisatiecodes, impliciete access tokens, client credentials en refresh tokens.
    \item Hydra is zeer uitgebreid en biedt ondersteuning voor geavanceerde autorisatiefuncties zoals dynamische toestemming verlening.
    \item Oathkeeper richt zich meer op toegangscontrole voor APIs en biedt mogelijk niet dezelfde diepgaande ondersteuning voor OAuth 2.0 als de andere twee.
  \end{itemize}
  
  \item Ondersteuning voor andere protocollen:
  \begin{itemize}
    \item Keycloak biedt ondersteuning voor OpenID Connect en SAML naast OAuth 2.0.
    \item Hydra richt zich voornamelijk op OAuth 2.0, maar biedt ondersteuning voor OpenID Connect.
    \item Oathkeeper is voornamelijk gericht op OAuth 2.0 en JWT verificatie.
  \end{itemize}
  
  \item Schaalbaarheid en prestaties:
  \begin{itemize}
    \item Keycloak kan in grootschalige implementaties prestatieproblemen ondervinden.
    \item Hydra is ontworpen met schaalbaarheid in gedachten en presteert over het algemeen goed in grote omgevingen.
    \item Oathkeeper is lichtgewicht en schaalbaar, maar het ontbreekt mogelijk aan enkele geavanceerde functies voor grote implementaties.
  \end{itemize}
  
  \item Aanpasbaarheid en uitbreidbaarheid:
  \begin{itemize}
    \item Keycloak biedt enige mate van aanpasbaarheid via plugins en themas.
    \item Hydra biedt uitgebreide aanpassingsmogelijkheden met behulp van plugins en aangepaste logica.
    \item Oathkeeper biedt mogelijkheden voor aanpassing, maar het is meer gericht op standaard API-gateway functionaliteit.
  \end{itemize}
  
  \item Documentatie en community ondersteuning:
  \begin{itemize}
    \item Keycloak heeft een grote community en uitgebreide documentatie.
    \item Hydra heeft ook een actieve community, maar de documentatie kan op sommige gebieden ontoereikend zijn.
    \item Oathkeeper heeft een groeiende community, maar de documentatie kan nog wat uitgebreider worden.
  \end{itemize}
  
  \item Beveiligingsfuncties:
  \begin{itemize}
    \item Alle drie de oplossingen bieden robuuste beveiligingsfuncties, waaronder ondersteuning voor multifactor authenticatie en integratie met externe identiteitsproviders.
  \end{itemize}
  
  \item Onderhoud en updates:
  \begin{itemize}
    \item Keycloak en Hydra worden actief onderhouden en bijgewerkt.
    \item Oathkeeper wordt ook regelmatig bijgewerkt, maar de ontwikkeling kan iets minder snel zijn dan bij de andere twee.
  \end{itemize}
  
  \item Beschikbaarheid van integraties:
  \begin{itemize}
    \item Alle drie de oplossingen bieden integraties met populaire frameworks, bibliotheken en platforms, hoewel de diepte van integraties kan variëren.
  \end{itemize}
  
  \item Kosten en licentie:
  \begin{itemize}
    \item Keycloak is gratis en open source.
    \item Hydra en Oathkeeper zijn ook open source, maar sommige functies kunnen onderworpen zijn aan commerciële licenties in bepaalde scenario's.
  \end{itemize}
\end{enumerate}

De zwaktes van veel identiteits- en toegangsbeheer oplossingen, waaronder Keycloak, zijn onder andere:

\begin{enumerate}
  \item Gebruikerservaring
  Een gebied waarop veel identiteits- en toegangsbeheer oplossingen tekortschieten, is de gebruikerservaring bij het inloggen en beheren van gebruikersaccounts. Vaak zijn de standaard authenticatie- en registratie flows te complex of niet goed afgestemd op de specifieke behoeften van een applicatie. Ontwerpen van intuïtieve en aanpasbare gebruikersinterfaces voor authenticatie en account beheer lijkt interessant.
  
  \item Schaalbaarheid en prestaties
  Bij het beheren van grote aantallen gebruikers en verzoeken kunnen sommige identiteits- en toegangsbeheer oplossingen te maken krijgen met prestatieproblemen en schaalbaarheid uitdagingen. Werken aan het optimaliseren van de prestaties en schaalbaarheid van een oplossing lijk een idee, bijvoorbeeld door gebruik te maken van caching, gegevens partitionering en schaalbare infrastructuur.
  
  \item Ondersteuning voor opkomende standaarden
  Hoewel veel oplossingen voldoen aan de gangbare standaarden zoals OAuth 2.0 en OpenID Connect, kunnen ze achterlopen bij het ondersteunen van opkomende standaarden en technologieën die relevant kunnen zijn voor bepaalde toepassingen. Investeren in het verkennen en implementeren van nieuwe standaarden zoals verifiable credentials en decentralized identifiers is een optie.
\end{enumerate}

Nu kunnen suggesties worden aangeboden voor verbeteringen en unieke functies in bestaande oplossingen zoals Keycloak, Hydra en Oathkeeper:

\begin{enumerate}
  \item Verbeterde schaalbaarheid en prestaties: Focus op het optimaliseren van de schaalbaarheid en prestaties van het systeem, zodat het gemakkelijk kan omgaan met grote aantallen gelijktijdige gebruikers en verzoeken. Gebruik bijvoorbeeld geavanceerde technieken zoals horizontale schaalbaarheid, caching en asynchrone verwerking om de prestaties te verbeteren.
  
  \item Verbeterde beveiligingsfuncties: Integreer geavanceerde beveiligingsfuncties zoals geavanceerde bedreiging detectie, geografische toegangscontrole, geavanceerde logging en auditing en machine learning-aangedreven anomalie detectie om de beveiliging van het systeem te verbeteren.
  
  \item Gebruiksvriendelijke configuratie en beheer: Zorg voor een intuïtieve gebruikersinterface voor configuratie en beheer, waardoor gebruikers gemakkelijk het systeem kunnen instellen en aanpassen aan hun specifieke behoeften. Bied ook uitgebreide documentatie en ondersteuning om gebruikers te helpen bij het configureren en beheren van het systeem.
  
  \item Verbeterde aanpasbaarheid en uitbreidbaarheid: Maak het systeem zeer aanpasbaar en uitbreidbaar, zodat gebruikers gemakkelijk functionaliteit kunnen toevoegen of aanpassen via plugins, extensies of aangepaste code. Bied een goed gedefinieerde API en ontwikkelplatform voor het bouwen van op maat gemaakte integraties en extensies.
  
  \item Geavanceerde autorisatie- en toestemming beheer: Ontwikkel geavanceerde autorisatie- en toestemming beheer functionaliteit, waaronder ondersteuning voor dynamische toestemming verlening, beleid gebaseerde toegangscontrole, contextuele autorisatie en geavanceerde gebruikersrollen en machtigingen.
  
  \item Verbeterde integraties met externe systemen: Bied uitgebreide integraties met externe systemen en services, waaronder identiteitsproviders, directory services, applicaties en API's. Zorg ervoor dat het systeem gemakkelijk kan integreren met populaire frameworks, bibliotheken en platforms die vaak worden gebruikt door ontwikkelaars.
\end{enumerate}

Door te focussen op deze gebieden kan een OAuth 2.0 authenticatiesysteem ontwikkeld worden dat zich onderscheidt van bestaande oplossingen door verbeterde prestaties, beveiliging, gebruiksgemak en aanpasbaarheid, waardoor het aantrekkelijk is voor gebruikers die op zoek zijn naar een geavanceerde en uitgebreide oplossing.
