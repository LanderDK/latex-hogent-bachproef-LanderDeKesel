%%=============================================================================
%% Methodologie
%%=============================================================================

\chapter{\IfLanguageName{dutch}{Methodologie}{Methodology}}%
\label{ch:methodologie}

%% TODO: In dit hoofstuk geef je een korte toelichting over hoe je te werk bent
%% gegaan. Verdeel je onderzoek in grote fasen, en licht in elke fase toe wat
%% de doelstelling was, welke deliverables daar uit gekomen zijn, en welke
%% onderzoeksmethoden je daarbij toegepast hebt. Verantwoord waarom je
%% op deze manier te werk gegaan bent.
%% 
%% Voorbeelden van zulke fasen zijn: literatuurstudie, opstellen van een
%% requirements-analyse, opstellen long-list (bij vergelijkende studie),
%% selectie van geschikte tools (bij vergelijkende studie, "short-list"),
%% opzetten testopstelling/PoC, uitvoeren testen en verzamelen
%% van resultaten, analyse van resultaten, ...
%%
%% !!!!! LET OP !!!!!
%%
%% Het is uitdrukkelijk NIET de bedoeling dat je het grootste deel van de corpus
%% van je bachelorproef in dit hoofstuk verwerkt! Dit hoofdstuk is eerder een
%% kort overzicht van je plan van aanpak.
%%
%% Maak voor elke fase (behalve het literatuuronderzoek) een NIEUW HOOFDSTUK aan
%% en geef het een gepaste titel.


\section{Vergelijkingscriteria Auth aanbieders}%
\label{sec:vergelijkingscriteria-auth-aanbieders}

\begin{enumerate}
  \item Hoofd functionaliteit: Wat zijn de belangrijkste functionaliteiten?
  \item Ontwikkelaars ervaring: Wat ervaren ontwikkelaars? Hoe gemakkelijk is het te gebruiken? (Documentatie, SDKs, integraties, \ldots)
  \item Integratie mogelijkheden: Hoe gemakkelijk integreert het met tech stacks, frameworks, gerelateerde producten, \ldots?
  \item Use Cases: Waarom en wanneer gebruik je de auth aanbieder het best?
\end{enumerate}


\section{Auth aanbieders vergelijken}%
\label{sec:auth-aanbieders-vergelijken}
Op basis van de eerder genoemde criteria, werden de volgende Docker Images verder besproken en vergeleken: Auth0, Okta, Firebase Authentication en Amazon Cognito.
\begin{enumerate}
  \item Hoofd functionaliteit:
  \begin{itemize}
    \item Auth0 biedt een uitgebreide ondersteuning voor diverse identiteitsproviders, waaronder sociale logins en bedrijfsverbindingen, waardoor gebruikers een naadloze en veelzijdige inlogervaring hebben. Het platform gaat verder dan standaard beveiligingsmaatregelen door multi-factor authenticatie (MFA) en anomaly detectie te bieden, waardoor extra bescherming wordt geboden tegen ongeautoriseerde toegangspogingen. Bovendien kunnen bedrijven de inlog- en aanmeldingspagina's aanpassen aan hun eigen huisstijl, wat bijdraagt aan een consistente merkervaring voor gebruikers. Met de rules engine van Auth0 kunnen organisaties op maat gemaakte authenticatie- en autorisatielogica implementeren, waardoor ze flexibel kunnen inspelen op specifieke beveiligingsbehoeften en zakelijke vereisten.
    \item Okta staat bekend om zijn uitgebreide oplossingen op het gebied van identity and access management (IAM), waaronder een scala aan functionaliteiten die organisaties helpen bij het beheren van gebruikersidentiteiten en toegangsrechten. Enkele van de prominente kenmerken zijn Single Sign-On (SSO), wat gebruikers in staat stelt om met één set inloggegevens toegang te krijgen tot meerdere applicaties, adaptieve authenticatie die dynamisch reageert op verschillende risiconiveaus, en multi-factor authenticatie voor extra beveiligingslagen. Daarnaast biedt Okta ook API Access Management om de beveiliging van API's te waarborgen, en levenscyclusbeheer functionaliteiten voor het efficiënt beheren van gebruikersaccounts en groepen binnen een organisatie.
    \item Firebase Authentication is een cruciaal onderdeel van de uitgebreide Firebase-suite. Het biedt een breed scala aan authenticatiemogelijkheden voor diverse platforms, waaronder web, mobiel en server. Met ondersteuning voor inloggen via e-mail/wachtwoord, sociale media en telefoonnummer verificatie, biedt het een veelzijdige oplossing voor gebruikersidentificatie. Bovendien kan Firebase Authentication naadloos worden geïntegreerd met andere services binnen het Firebase-ecosysteem, waardoor ontwikkelaars een gestroomlijnde en consistente ontwikkelingservaring krijgen.
    \item Amazon Cognito is een uitgebreide identiteitsservice die wordt beheerd door AWS (Amazon Web Services). Het biedt een scala aan functionaliteiten, waaronder gebruikersregistratie en -aanmelding, multi-factor authenticatie en veilige toegangscontrole. Bovendien integreert het naadloos met AWS Identity and Access Management (IAM), waardoor gedetailleerde controle over toegangsrechten mogelijk is.
  \end{itemize}
  
  \item Ontwikkelaars ervaring:
  \begin{itemize}
    \item Auth0 heeft goed gedocumenteerde API's en SDK's voor verschillende platformen en uitgebreide reeks functies voor ontwikkelaars, inclusief hooks voor het aanpassen van workflows.
    \item Okta heeft goed gedocumenteerde API's en SDK's voor makkelijke integratie mogelijk te maken omdat het kant-en-klare integraties biedt met populaire applicaties.
    \item Firebase Authentication biedt naadloze integratie met andere Firebase-services, alsook vereenvoudigde SDK's voor populaire platformen.
    \item Amazon heeft goed gedocumenteerde API's en SDK's voor verschillende programmeertalen met naadloze integratie met andere AWS-services.
  \end{itemize}
  
  \item Integratie mogelijkheden:
  \begin{itemize}
    \item Auth0 maakt moeiteloze integratie met diverse technologieën en frameworks mogelijk. Het biedt uitgebreide ondersteuning voor Single Sign-On (SSO) voor alle applicaties, waardoor gebruikers moeiteloos toegang krijgen zonder herhaaldelijk in te loggen.
    \item Okta kan goed worden geïntegreerd met verschillende cloudservices en bedrijfsapplicaties. Het biedt robuuste ondersteuning voor SSO en toegangsbeheer.
    \item Firebase Authentication is ontworpen voor integratie met Firebase-aangedreven applicaties. Geschikt voor projecten gebouwd op het Firebase-platform.
    \item Amazon Cognito integreert soepel met andere AWS-services en cloud-native applicaties. Geschikt voor projecten binnen het AWS-ecosysteem.
  \end{itemize}
  
  \item Use Cases:
  \begin{itemize}
    \item Auth0 is geschikt voor een breed scala aan toepassingen, van kleine projecten tot grote bedrijfsoplossingen.
    \item Okta is zeer geschikt voor ondernemingen die een uitgebreide IAM-oplossing nodig hebben met de nadruk op beveiliging en compliance.
    \item Firebase Authentication is ideaal voor ontwikkelaars die aan projecten binnen het Firebase-ecosysteem werken.
    \item Amazon Cognito is ideaal voor applicaties die worden gehost op AWS en biedt schaalbaarheid en nauwe integratie met andere AWS-services.
  \end{itemize}
\end{enumerate}



\section{Vergelijkingscriteria OAuth 2.0 Docker Image's}%
\label{sec:vergelijkingscriteria-oauth2-docker-image}

\begin{enumerate}
  \item Gebruiksgemak en configuratie: Hoe gemakkelijk is het om de Docker-image in te stellen en te configureren voor je specifieke gebruiksscenario? Worden er duidelijke instructies geleverd?

  \item Ondersteunde OAuth 2.0-functies: Controleer welke functies van het OAuth 2.0 protocol worden ondersteund door de verschillende implementaties. Dit kan onder meer autorisatiecodes, impliciete access tokens, client credentials en refresh tokens omvatten.
  
  \item Ondersteuning voor andere protocollen: Sommige auth server implementaties ondersteunen naast OAuth 2.0 ook andere protocollen zoals OpenID Connect. Het kan handig zijn om te beoordelen welke protocollen worden ondersteund als je behoeften hebt die verder gaan dan alleen OAuth 2.0.
  
  \item Schaalbaarheid en prestaties: Hoe schaalbaar is de auth server? Kan het omgaan met een groot aantal gelijktijdige gebruikers en verzoeken? Zijn er prestatie benchmarks beschikbaar?
  
  \item Aanpasbaarheid en uitbreidbaarheid: Kun je de functionaliteit van de auth server aanpassen of uitbreiden met behulp van plugins of aangepaste code?
  
  \item Documentatie en community ondersteuning: Is er uitgebreide documentatie beschikbaar voor de Docker-image en de bijbehorende auth server? Is er een actieve community waar je vragen kunt stellen en ondersteuning kunt krijgen?
  
  \item Beveiligingsfuncties: Welke beveiligingsfuncties biedt de auth server? Wordt er bijvoorbeeld ondersteuning geboden voor multifactor authenticatie, JWT verificatie of integratie met externe identiteitsproviders?
  
  \item Onderhoud en updates: Wordt de Docker-image regelmatig bijgewerkt met bug fixes en beveiliging patches? Hoe actief is de ontwikkeling van de auth server?
  
  \item Beschikbaarheid van integraties: Controleer of de auth server integraties biedt met populaire frameworks, bibliotheken en platforms die je gebruikt in je applicatie stack.
  
  \item Kosten en licentie: Sommige auth server implementaties zijn gratis en open source, terwijl andere een commercieel licentiemodel hebben. Overweeg welke kosten er zijn verbonden aan het gebruik van de verschillende implementaties en of de licentievoorwaarden aansluiten bij je behoeften.
\end{enumerate}


\section{Long list}%
\label{sec:long-list}
\begin{table}[htbp]
  \centering
  \caption{OAuth 2.0-authenticatieservers en Docker-image beschikbaarheid}
  \label{tab:oauth_servers}
  \begin{adjustbox}{width=1\textwidth}
  \begin{tabular}{@{}llll@{}}
    \toprule
    Naam          & Website                               & Beschrijving                                                                   & Docker-image beschikbaar \\ \midrule
    Keycloak      & \texttt{https://www.keycloak.org/}     & Open source identiteits- en toegangsbeheer voor moderne applicaties en services. & Ja                        \\
    Hydra         & \texttt{https://www.ory.sh/hydra/}     & OAuth 2.0 en OpenID Connect-server met krachtige functies voor authenticatie en autorisatie. & Ja                        \\
    Oathkeeper & \texttt{https://www.ory.sh/oathkeeper/} & Identity \& Access Proxy (IAP) gebouwd op top van Ory Hydra en Ory Keto. & Ja                        \\
    Gluu          & \texttt{https://www.gluu.org/}         & Open source IAM-platform voor web- en mobiele applicaties.                   & Ja (community images)    \\
    Apereo CAS    & \texttt{https://apereo.github.io/cas/} & Central Authentication Service (CAS) voor authenticatie en autorisatie.      & Ja                        \\
    Dex           & \texttt{https://dexidp.io/}            & Open source OIDC-provider met LDAP-ondersteuning.                             & Ja                        \\
    FusionAuth    & \texttt{https://fusionauth.io/}        & Identity and access management voor developers.                               & Ja                        \\
    LemonLDAP::NG & \texttt{https://lemonldap-ng.org/}     & Open source toegangsbeheer voor webapplicaties.                               & Ja                        \\
    Keycloak Gatekeeper & \texttt{https://www.keycloak.org/docs/latest/securing\_apps/} & Een authenticatie-gateway die werkt met Keycloak.                     & Ja                        \\
    IdentityServer & \texttt{https://identityserver.io/}    & OpenID Connect- en OAuth 2.0-protocolserver voor ASP.NET Core.               & Nee                       \\
    Apache Oltu   & \texttt{https://oltu.apache.org/}      & OAuth 2.0-bibliotheken voor Java.                                             & Nee                       \\
    UAA           & \texttt{https://github.com/cloudfoundry/uaa} & Open source identiteitsbeheerservice voor Cloud Foundry.                & Nee                       \\
    Okta          & \texttt{https://www.okta.com/}         & Identity Cloud-service met ondersteuning voor OAuth 2.0 en OpenID Connect.    & Nee                       \\
    Auth0         & \texttt{https://auth0.com/}            & Identity-platform voor ontwikkelaars met OAuth 2.0 en OpenID Connect-ondersteuning. & Nee                       \\
    AWS Cognito   & \texttt{https://aws.amazon.com/cognito/} & Identity-service van Amazon Web Services met ondersteuning voor OAuth 2.0 en OpenID Connect. & Nee                       \\ \bottomrule
  \end{tabular}
  \end{adjustbox}
\end{table}


\section{Short list}%
\label{sec:short-list}
Hier is een shortlist van de meest veelbelovende alternatieven:
\begin{enumerate}
    \item Keycloak
    \item Hydra
    \item Oathkeeper
\end{enumerate}

Deze drie alternatieven hebben het hoogste totaal aantal voldane vereisten en hebben ook Docker-images beschikbaar.

Voor de proof-of-concept kan worden voorgesteld om één van deze drie alternatieven te kiezen en een eenvoudige implementatie op te zetten om een basaal scenario te testen. Dit kan bijvoorbeeld inhouden:

\begin{itemize}
    \item Het opzetten van een OAuth 2.0-authenticatieserver met het gekozen alternatief.
    \item Het implementeren van een eenvoudige client-applicatie die de authenticatie via OAuth 2.0 integreert.
    \item Het uitvoeren van enkele basisauthenticatie- en autorisatietests om te controleren of het alternatief voldoet aan de vereisten zoals gebruikersbeheer, toegangscontrole en veiligheid.
\end{itemize}

Hierdoor kan de werking van het gekozen alternatief beter begrepen worden en bepalen of het geschikt is voor verdere implementatie in je project.


\section{Short list alternatieven testen}
\label{sec:short-list-alternatieven-testen}
In deze sectie wordt er verder verdiept in de drie alternatieven, namelijk Keycloak, Hydra en Oathkeeper.
Het doel is de alternatieven van de short list te testen en te evalueren op basis van de eerder vastgestelde criteria. Indien er pijnpunten of tekortkomingen worden vastgesteld, zullen deze worden opgesomd en zullen er suggesties worden gegeven voor mogelijke verbeteringen.

\subsection{Keycloak}%
\label{subsec:keycloak}
\subsubsection{Stappenplan voor het opzetten van Keycloak als Docker Image}%
\label{subsubsec:keycloak-setup}
Zojuist werd de ervaring beschreven bij het opzetten en gebruiken van Keycloak, in deze sectie wordt er stap voor stap uitgelegd wat er effectief gedaan werd.\\\\
\textbf{Stap 1: Installatievereisten}\\
Zorg ervoor dat Docker op uw systeem geïnstalleerd is. U kunt de installatiehandleidingen voor Docker volgen op \texttt{https://docs.docker.com/get-docker/}.\\\\
\textbf{Stap 2: Keycloak Docker Image Downloaden}\\
Download het Keycloak Docker-image vanuit de Docker Hub. Gebruik het volgende commando:
\begin{lstlisting}[language=bash]
docker pull quay.io/keycloak/keycloak:latest
\end{lstlisting}
\textbf{Stap 3: Keycloak Container Starten}\\
Start een Keycloak-container met het volgende commando. Dit zal een container starten met de standaard admin-gebruiker en wachtwoord ingesteld
\begin{lstlisting}[language=bash]
docker run -d --name keycloak \
    -p 8080:8080 \
    -e KEYCLOAK_USER=admin \
    -e KEYCLOAK_PASSWORD=admin \
    quay.io/keycloak/keycloak:latest
\end{lstlisting}
\textbf{Stap 4: Toegang tot Keycloak}\\
Open een webbrowser en ga naar \texttt{http://localhost:8080}. U zou het Keycloak-welkomstscherm moeten zien.\\\\
\textbf{Stap 5: Inloggen bij de Keycloak Admin Console}\\
Klik op de link ``Administration Console'' en log in met de gebruikersnaam \texttt{admin} en het wachtwoord \texttt{admin} die u in stap 3 hebt ingesteld.\\\\
\textbf{Stap 6: Configureren van een Realm}\\
Nadat u bent ingelogd, kunt u beginnen met het configureren van uw realm. Een realm vertegenwoordigt een beveiligingsdomein in Keycloak. Volg deze stappen om een nieuwe realm aan te maken:
\begin{enumerate}
    \item Klik op het dropdown-menu rechtsboven naast het Keycloak-logo en selecteer ``Add realm''.
    \item Voer een naam in voor uw realm en klik op ``Create''.
\end{enumerate}
\textbf{Stap 7: Configureren van een Client}\\
Een client vertegenwoordigt een applicatie waarvoor u Keycloak als identiteits- en toegangsbeheeroplossing wilt gebruiken. Volg deze stappen om een nieuwe client aan te maken:
\begin{enumerate}
    \item Selecteer uw aangemaakte realm.
    \item Klik op ``Clients'' in het linkermenu.
    \item Klik op ``Create'' om een nieuwe client toe te voegen.
    \item Voer een client ID in, bijvoorbeeld \texttt{my-client}, en selecteer het protocol dat uw applicatie gebruikt (bijvoorbeeld \texttt{openid-connect}).
    \item Klik op ``Save'' om de client aan te maken.
\end{enumerate}
\textbf{Stap 8: Configureren van Gebruikers en Rollen}\\
Voeg gebruikers en rollen toe aan uw realm om toegang te verlenen aan uw applicaties:
\begin{enumerate}
    \item Klik op ``Users'' in het linkermenu.
    \item Klik op ``Add user'' om een nieuwe gebruiker toe te voegen.
    \item Voer de gebruikersgegevens in en klik op ``Save''.
    \item Ga naar het tabblad ``Credentials'' om een wachtwoord voor de gebruiker in te stellen.
    \item Klik op ``Roles'' in het linkermenu om rollen toe te voegen die u kunt toewijzen aan gebruikers en clients.
\end{enumerate}
\textbf{Conclusie}\\
Met deze stappen heeft u een functionele Keycloak-installatie draaien in een Docker-container, klaar voor gebruik in uw applicaties. U kunt nu verdergaan met het configureren van meer geavanceerde instellingen en integraties op basis van uw specifieke behoeften.


\subsection{Hydra}%
\label{subsec:hydra}
\subsubsection{Stappenplan voor het opzetten van Hydra als Docker Image}%
\label{subsubsec:hydra-setup}
Zojuist werd de ervaring beschreven bij het opzetten en gebruiken van Hydra, in deze sectie wordt er stap voor stap uitgelegd wat er effectief gedaan werd.\\\\
\textbf{Stap 1: Installatievereisten}\\
Zorg ervoor dat Docker en Docker Compose op uw systeem geïnstalleerd zijn. U kunt de installatiehandleidingen voor Docker en Docker Compose volgen op\\ \texttt{https://docs.docker.com/get-docker/} en \texttt{https://docs.docker.com/compose/install/}.\\\\
\textbf{Stap 2: Hydra Docker Image Downloaden}\\
Download het Hydra Docker-image vanuit de Docker Hub. Gebruik het volgende commando:
\begin{lstlisting}[language=bash]
  docker pull oryd/hydra:latest
\end{lstlisting}
\textbf{Stap 3: Hydra Configureren}\\
Voordat we de Hydra-container starten, moeten we enkele configuratiebestanden aanmaken. Maak een nieuw bestand genaamd \texttt{hydra-config.yaml} met de volgende inhoud:
\begin{lstlisting}[language=bash]
  dsn: memory
  log:
    level: info
    format: text
  serve:
    public:
      port: 4444
    admin:
      port: 4445
  urls:
    self:
      public: http://localhost:4444/
      admin: http://localhost:4445/
\end{lstlisting}
\textbf{Stap 4: Hydra Container Starten}\\
Start een Hydra-container met het volgende commando. Dit zal de configuratie laden vanuit het bestand dat we zojuist hebben aangemaakt.
\begin{lstlisting}[language=bash]
docker run -d --name hydra \
    -p 4444:4444 -p 4445:4445 \
    -v $(pwd)/hydra-config.yaml:/etc/hydra/config.yaml \
    oryd/hydra:latest serve all --config /etc/hydra/config.yaml
\end{lstlisting}
\textbf{Stap 5: Hydra Initialiseren}\\
Initialiseer Hydra door de volgende commando's uit te voeren. Dit zal een admin client aanmaken die nodig is voor de verdere configuratie.
\begin{lstlisting}[language=bash]
docker exec -it hydra \
    hydra clients create \
    --endpoint http://localhost:4445 \
    --id my-client \
    --secret secret \
    --grant-types authorization_code,refresh_token,client_credentials \
    --response-types code,id_token \
    --scope openid,offline_access \
    --callbacks http://localhost:5555/callback
\end{lstlisting}
\textbf{Stap 6: Toegang tot Hydra}\\
Hydra is nu klaar voor gebruik. U kunt toegang krijgen tot de admin endpoints via \texttt{http://localhost:4445} en de public endpoints via \texttt{http://localhost:4444}.\\\\
\textbf{Stap 7: Configureren van OAuth 2.0 Clients}\\
Voeg nieuwe clients toe via de Hydra admin API.\@ U kunt een nieuwe client configureren met de volgende commando's:
\begin{lstlisting}[language=bash]
docker exec -it hydra \
    hydra clients create \
    --endpoint http://localhost:4445 \
    --id another-client \
    --secret another-secret \
    --grant-types authorization_code,refresh_token,client_credentials \
    --response-types code,id_token \
    --scope openid,offline_access \
    --callbacks http://localhost:5555/callback
\end{lstlisting}
\textbf{Stap 8: Configureren van Policies en Consent App}\\
Hydra vereist een consent app voor het beheren van gebruikersconsent. U kunt een eenvoudige consent app opzetten of de bestaande Hydra consent app gebruiken. Raadpleeg de officiële documentatie voor gedetailleerde instructies over het configureren van de consent app.
\begin{lstlisting}[language=bash]
docker run -d --name hydra-consent \
    -p 5555:3000 \
    oryd/hydra-login-consent-node:v1.10.6
\end{lstlisting}
\textbf{Stap 9: Hydra Integreren met Externe Identiteitsproviders}\\
Hydra kan worden geïntegreerd met externe identiteitsproviders zoals Google, Facebook, etc. Deze configuratie hangt af van de specifieke provider en vereist het opzetten van de juiste credentials en redirect URLs. Raadpleeg de Hydra documentatie voor gedetailleerde stappen.\\
\textbf{Conclusie}\\
Met deze stappen zou u een werkende Hydra setup moeten hebben die klaar is voor gebruik in uw ontwikkelomgeving.


\subsection{Oathkeeper}%
\label{subsec:oathkeeper}
\subsubsection{Stappenplan voor het opzetten van Oathkeeper als Docker Image}%
\label{subsubsec:oathkeeper-setup}
Zojuist werd de ervaring beschreven bij het opzetten en gebruiken van Oathkeeper, in deze sectie wordt er stap voor stap uitgelegd wat er effectief gedaan werd.\\\\
\textbf{Stap 1: Installatievereisten}\\
Zorg ervoor dat Docker en Docker Compose op uw systeem geïnstalleerd zijn. U kunt de installatiehandleidingen voor Docker en Docker Compose volgen op\\ \texttt{https://docs.docker.com/get-docker/} en \texttt{https://docs.docker.com/compose/install/}.\\\\
\textbf{Stap 2: Oathkeeper Docker Image Downloaden}\\
Download het Oathkeeper Docker-image vanuit de Docker Hub. Gebruik het volgende commando:
\begin{lstlisting}[language=bash]
docker pull oryd/oathkeeper:latest
\end{lstlisting}
\textbf{Stap 3: Oathkeeper Configureren}\\
Voordat we de Oathkeeper-container starten, moeten we enkele configuratiebestanden aanmaken. Maak een nieuw bestand genaamd \texttt{oathkeeper-config.yaml} met de volgende inhoud:
\begin{lstlisting}[language=bash]
authenticators:
  noop:
    enabled: true
authorizers:
  allow:
    enabled: true
mutators:
  noop:
    enabled: true
access_rules:
  repositories:
    - file:///etc/oathkeeper/rules/rules.json
serve:
  proxy:
    port: 4455
  api:
    port: 4456
log:
  level: debug
\end{lstlisting}
Maak vervolgens een regelsbestand \texttt{rules.json} aan met enkele voorbeeldregels:
\begin{lstlisting}[language=bash]
[
  {
    "id": "example-rule",
    "upstream": {
      "url": "http://example.com"
    },
    "match": {
      "url": "http://localhost:4455/<.*>",
      "methods": ["GET"]
    },
    "authenticators": [
      {
        "handler": "noop"
      }
    ],
    "authorizer": {
      "handler": "allow"
    },
    "mutators": [
      {
        "handler": "noop"
      }
    ]
  }
]
\end{lstlisting}
\textbf{Stap 4: Oathkeeper Container Starten}\\
Start een Oathkeeper-container met het volgende commando. Dit zal de configuratie laden vanuit de bestanden die we zojuist hebben aangemaakt.
\begin{lstlisting}[language=bash]
docker run -d --name oathkeeper \
    -p 4455:4455 -p 4456:4456 \
    -v $(pwd)/oathkeeper-config.yaml:/etc/oathkeeper/oathkeeper.yaml \
    -v $(pwd)/rules.json:/etc/oathkeeper/rules/rules.json \
    oryd/oathkeeper:latest serve --config /etc/oathkeeper/oathkeeper.yaml
\end{lstlisting}
\textbf{Stap 5: Toegang tot Oathkeeper}\\
Oathkeeper is nu klaar voor gebruik. U kunt toegang krijgen tot de proxy endpoints via \texttt{http://localhost:4455} en de API endpoints via \texttt{http://localhost:4456}.\\\\
\textbf{Stap 6: Configureren van Access Rules}\\
Voeg nieuwe toegangsregels toe door het \texttt{rules.json} bestand te bewerken en de container opnieuw te starten of de configuratie opnieuw te laden via de Oathkeeper API.\@
\begin{lstlisting}[language=bash]
docker exec oathkeeper oathkeeper api --endpoint http://localhost:4456 rules create \
    -f /etc/oathkeeper/rules/rules.json
\end{lstlisting}
\textbf{Stap 7: Integreren met Identiteitsproviders}\\
Oathkeeper kan worden geïntegreerd met identiteitsproviders zoals Hydra, Keycloak, etc. Dit vereist het configureren van de juiste authenticators en authorizers in het configuratiebestand. Raadpleeg de officiële Oathkeeper documentatie voor gedetailleerde instructies over het configureren van integraties.\\\\
\textbf{Stap 8: Monitoren en Loggen}\\
Gebruik de ingebouwde loggingfunctionaliteit om toegangspogingen en beleidsbeslissingen te monitoren. De logniveau kan worden aangepast in het configuratiebestand (\texttt{oathkeeper-config.yaml}) door de \texttt{log.level} parameter aan te passen.\\\\
\textbf{Conclusie}\\
Met deze stappen zou u een werkende Oathkeeper setup moeten hebben die klaar is voor gebruik in uw ontwikkelomgeving.


\section{Docker Images vergelijken}%
\label{sec:docker-images-vergelijken}
Op basis van de eerder genoemde criteria, werden de volgende Docker Images verder besproken en vergeleken: Keycloak, Hydra en Oathkeeper.
De bedoeling van deze sectie is om nog eens op te sommen wat elk van de alternatieven in de short list goed of minder doen, per vergelijkingscriterium.
Dit zodat men weet welk alternatief het best past bij zijn of haar use case.

\begin{enumerate}
  \item Gebruiksgemak en configuratie:
  \begin{itemize}
    \item Keycloak wordt over het algemeen beschouwd als gebruiksvriendelijk met een intuïtieve gebruikersinterface voor configuratie.
    \item Hydra vereist wat meer configuratie, vooral als het gaat om het opzetten van complexe autorisatie flows.
    \item Oathkeeper kan uitdagender zijn om te configureren vanwege de focus op API-gateway functionaliteit, hoewel het krachtige mogelijkheden biedt.
  \end{itemize}
  
  \item Ondersteunde OAuth 2.0-functies:
  \begin{itemize}
    \item Keycloak biedt een breed scala aan functies, waaronder autorisatiecodes, impliciete access tokens, client credentials en refresh tokens.
    \item Hydra is zeer uitgebreid en biedt ondersteuning voor geavanceerde autorisatiefuncties zoals dynamische toestemming verlening.
    \item Oathkeeper richt zich meer op toegangscontrole voor APIs en biedt mogelijk niet dezelfde diepgaande ondersteuning voor OAuth 2.0 als de andere twee.
  \end{itemize}
  
  \item Ondersteuning voor andere protocollen:
  \begin{itemize}
    \item Keycloak biedt ondersteuning voor OpenID Connect en SAML naast OAuth 2.0.
    \item Hydra richt zich voornamelijk op OAuth 2.0, maar biedt ondersteuning voor OpenID Connect.
    \item Oathkeeper is voornamelijk gericht op OAuth 2.0 en JWT verificatie.
  \end{itemize}
  
  \item Schaalbaarheid en prestaties:
  \begin{itemize}
    \item Keycloak is een robuust en schaalbaar platform dat wordt gebruikt in verschillende omgevingen, variërend van kleine tot grote ondernemingen. Het biedt clusteringmogelijkheden voor schaalbaarheid en kan worden geconfigureerd voor hoge beschikbaarheid. De prestaties zijn over het algemeen goed, maar bij zeer grote implementaties kan het nodig zijn om de infrastructuur zorgvuldig te plannen en te schalen om optimale prestaties te garanderen.
    \item Hydra is ook ontworpen met schaalbaarheid in gedachten en wordt vaak gebruikt in moderne cloudomgevingen. Het is gebouwd met behulp van Go, wat bekend staat om zijn goede prestaties en efficiënt gebruik van systeembronnen. Hydra kan worden geconfigureerd voor clustering en hoge beschikbaarheid om te voldoen aan de behoeften van grootschalige implementaties. Over het algemeen presteert Hydra goed in termen van snelheid en schaalbaarheid.
    \item Oathkeeper is ontworpen met schaalbaarheid in gedachten en kan worden gebruikt in cloud-native omgevingen waar microservices worden gebruikt. Het is lichtgewicht en kan horizontaal worden geschaald om te voldoen aan de vereisten van groeiende applicaties en services. De prestaties kunnen over het algemeen goed zijn, maar net als bij Keycloak is het belangrijk om de infrastructuur en configuratie te optimaliseren voor maximale efficiëntie.
  \end{itemize}
  
  \item Aanpasbaarheid en uitbreidbaarheid:
  \begin{itemize}
    \item Keycloak biedt een krachtig extensiemodel waarmee ontwikkelaars functionaliteit kunnen aanpassen en uitbreiden met behulp van plugins en aangepaste code. Het ondersteunt verschillende soorten extensies, waaronder authenticators, thema's, providers en meer. Met behulp van het JBoss Modules-systeem kunnen externe modules worden gemaakt en ingezet om aanvullende functionaliteit toe te voegen. Het aanpassen van Keycloak is over het algemeen goed gedocumenteerd en er zijn veel voorbeelden en community-ondersteuning beschikbaar.
    \item Hydra heeft een modulair ontwerp dat aanpassingen en uitbreidingen mogelijk maakt, zij het met een iets andere benadering dan Keycloak en Oathkeeper. Het biedt ondersteuning voor plugins voor bepaalde functionaliteit, zoals authenticators en authorizers. Hydra is gebouwd met behulp van Go, wat een krachtige en efficiënte programmeertaal is, maar het kan zijn dat ontwikkelaars die niet bekend zijn met Go en de Hydra-architectuur meer moeite hebben om aanpassingen te maken.
    \item Oathkeeper biedt ook mogelijkheden voor aanpassing en uitbreiding, zij het in mindere mate dan Keycloak. Het ondersteunt plugins voor authenticatie- en autorisatiestrategieën, waardoor ontwikkelaars kunnen aanpassen hoe toegangscontrole wordt afgedwongen. Hoewel de documentatie niet zo uitgebreid is als die van Keycloak, biedt Oathkeeper wel enige ondersteuning voor aangepaste code en integratie met externe systemen.
  \end{itemize}
  
  \item Documentatie en community ondersteuning:
  \begin{itemize}
    \item Keycloak heeft uitgebreide documentatie beschikbaar op hun officiële website, inclusief handleidingen, tutorials en referentiedocumentatie. Ze bieden ook Docker-images voor eenvoudige implementatie en hebben een actieve community op hun forums en andere platforms zoals Stack Overflow, waar ontwikkelaars vragen kunnen stellen en ondersteuning kunnen krijgen. De community rond Keycloak is over het algemeen behoorlijk actief en behulpzaam.
    \item Hydra wordt onderhouden door ORY, en ze bieden documentatie op hun officiële website, inclusief handleidingen, configuratiegidsen en API-referenties, maar de documentatie kan op sommige gebieden ontoereikend zijn. Ze hebben Docker-images beschikbaar voor eenvoudige implementatie en een actieve community op hun forums en GitHub-repository, waar ontwikkelaars vragen kunnen stellen en problemen kunnen melden. De community rond Hydra is over het algemeen behoorlijk betrokken en biedt goede ondersteuning aan gebruikers.
    \item Oathkeeper heeft ook documentatie beschikbaar op hun officiële website, inclusief installatiehandleidingen en referentiedocumentatie voor configuratie en gebruik. Hoewel de documentatie mogelijk niet zo uitgebreid is als die van Keycloak maar wel uitgebreider dan die van Hydra, bieden ze wel een redelijke basis voor het aan de slag gaan met het systeem. Oathkeeper heeft ook een gemeenschap rond hun GitHub-repository en andere kanalen waar gebruikers vragen kunnen stellen en hulp kunnen krijgen van andere ontwikkelaars.
  \end{itemize}
  
  \item Beveiligingsfuncties:
  \begin{itemize}
    \item Keycloak biedt ondersteuning voor verschillende vormen van MFA, waaronder One-Time Passwords (OTP), SMS-verificatie, e-mailverificatie, biometrische authenticatie en meer. Het kan JWT's (JSON Web Tokens) verifiëren die worden gebruikt voor authenticatie en autorisatie. Ook is er ondersteuning voor integratie met externe identiteitsproviders, waaronder SAML 2.0 Identity Providers, OAuth 2.0 Authorization Servers en LDAP-systemen.
    \item Hydra zelf biedt geen ingebouwde MFA-functionaliteit, maar kan worden geïntegreerd met externe systemen voor MFA. Het ondersteunt verificatie van JWT's voor het afdwingen van toegangscontrole. Ook kan het worden geconfigureerd om samen te werken met externe identiteitsproviders voor authenticatie en autorisatie.
    \item Oathkeeper zelf biedt geen ingebouwde MFA-functionaliteit, maar kan worden geïntegreerd met externe systemen voor MFA. Het ondersteunt verificatie van JWT's voor het afdwingen van toegangscontrole. Ook kan het worden geconfigureerd om samen te werken met externe identiteitsproviders voor authenticatie en autorisatie.
  \end{itemize}
  
  \item Onderhoud en updates:
  \begin{itemize}
    \item Keycloak wordt wekelijks bijgewerkt met bug fixes en beveiligingspatches, en er zijn regelmatig nieuwe releases met nieuwe functies en verbeteringen. De ontwikkeling van Keycloak is actief en er is een roadmap beschikbaar waarin toekomstige functies en verbeteringen worden beschreven.
    \item Hydra wordt meestal maandelijks bijgewerkt met bug fixes en beveiligingspatches, en er zijn regelmatig nieuwe releases met nieuwe functies en verbeteringen. De ontwikkeling van Hydra is actief en er is een roadmap en changelog beschikbaar waarin toekomstige functies en verbeteringen worden beschreven.
    \item Oathkeeper werd maandelijks bijgewerkt, maar de laatste tijd zijn er minder updates geweest. De ontwikkeling van Oathkeeper lijkt minder actief te zijn dan die van Keycloak en Hydra, maar er zijn nog steeds regelmatig bug fixes en beveiligingspatches beschikbaar. Er is een roadmap en changelog beschikbaar waarin toekomstige functies en verbeteringen worden beschreven.
  \end{itemize}
  
  \item Beschikbaarheid van integraties:
  \begin{itemize}
    \item Keycloak heeft een diepe integratie met Java, Spring Framework, Node.js en Angular. Hoewel er geen officiële React-adapter is, zijn er community-bibliotheken beschikbaar voor integratie met React-toepassingen.
    \item Hydra is gebouwd in Go en biedt uitstekende integratie met Go-toepassingen. Hoewel er geen officiële Node.js-adapter is, zijn er community-bibliotheken beschikbaar voor integratie met Node.js-toepassingen. Hydra kan worden geïntegreerd met vrijwel elk platform via HTTP, waardoor het geschikt is voor verschillende programmeertalen en frameworks.
    \item Oathkeeper biedt integraties op laag niveau met verschillende programmeertalen en frameworks, maar het heeft geen specifieke adapters of integraties met populaire frameworks zoals Keycloak. Het kan worden gebruikt in combinatie met verschillende programmeertalen en frameworks via HTTP, waardoor ontwikkelaars vrij zijn om het te integreren zoals ze willen.
  \end{itemize}
  
  \item Kosten en licentie:
  \begin{itemize}
    \item Keycloak is een gratis en open-source project, wat betekent dat het vrij te gebruiken is zonder kosten voor licenties. Keycloak wordt uitgebracht onder de Apache License 2.0, wat een permissieve open-source licentie is die gebruikers veel vrijheid biedt bij het gebruik en distribueren van de software. Het staat gebruikers toe om de software aan te passen en te distribueren, zelfs in commerciële contexten, zolang ze voldoen aan de voorwaarden van de licentie.
    \item Hydra is ook een gratis en open-source project, dus er zijn geen kosten verbonden aan het gebruik ervan. Hydra wordt uitgebracht onder de Apache License 2.0, vergelijkbaar met Keycloak. Dit betekent dat het vrij te gebruiken, aan te passen en te distribueren is onder de voorwaarden van de licentie.
    \item Oathkeeper is ook een open-source project en is gratis te gebruiken zonder kosten voor licenties. Oathkeeper wordt uitgebracht onder de Apache License 2.0, net als Keycloak en Hydra. Het biedt dezelfde vrijheden en voorwaarden voor gebruik als de andere twee systemen.
  \end{itemize}
\end{enumerate}

Op basis van de informatie die we hebben besproken, kunnen we enkele potentiële zwakke punten identificeren voor elk van de auth server-implementaties:

\begin{enumerate}[label=\textbf{\arabic*.}]
    \item \textbf{Keycloak}:
    \begin{itemize}
        \item \textbf{Resource-intensief}: Voor grootschalige implementaties kan Keycloak behoorlijk resource-intensief zijn, vooral als het gaat om geheugen- en CPU-gebruik.
    \end{itemize}

    \item \textbf{Hydra}:
    \begin{itemize}
        \item \textbf{Steilere leercurve}: Hydra kan een steilere leercurve hebben vanwege zijn meer modulaire en programmeergerichte aanpak. Dit kan het implementeren en aanpassen ervan bemoeilijken voor gebruikers zonder diepgaande technische kennis.
        \item \textbf{Minder integraties}: Hoewel Hydra flexibel is, biedt het mogelijk minder integraties met populaire frameworks en bibliotheken in vergelijking met Keycloak, vooral voor specifieke programmeertalen en platforms. Dit komt omdat Hydra een meer modulaire en programmeergerichte aanpak, wat betekent dat ontwikkelaars mogelijk meer maatwerk moeten verrichten om integraties te implementeren met specifieke frameworks of bibliotheken. Dit kan leiden tot minder standaardintegraties in vergelijking met een meer allesomvattende oplossing zoals Keycloak.
    \end{itemize}
    
    \item \textbf{Oathkeeper}:
    \begin{itemize}
        \item \textbf{Beperkte functionaliteit}: Oathkeeper biedt mogelijk niet dezelfde uitgebreide functionaliteit als Keycloak, vooral op het gebied van identiteits- en toegangsbeheer. Het kan meer gericht zijn op specifieke use-cases en vereist mogelijk aanvullende tools of integraties voor een complete oplossing.
        \item \textbf{Minder community ondersteuning}: Oathkeeper heeft mogelijk minder community ondersteuning in vergelijking met Keycloak, wat kan resulteren in minder beschikbare bronnen en hulp bij problemen of vragen.
    \end{itemize}
\end{enumerate}

Het is belangrijk op te merken dat deze zwakke punten relatief zijn en afhankelijk zijn van de specifieke behoeften, vaardigheden en vereisten van een project. Wat voor de ene gebruiker als een zwak punt wordt beschouwd, kan voor een andere gebruiker geen probleem zijn. Het is altijd raadzaam om een grondige evaluatie te maken op basis van uw specifieke situatie voordat u een beslissing neemt over welke auth server-implementatie het meest geschikt is voor uw project.

\subsection{Verbeteringen voor Bestaande Systemen}%
\label{subsec:verbeteringen-voor-bestaande-systemen}
\subsubsection{Verbeteringen voor Keycloak:}%
\label{subsubsec:verbeteringen-voor-keycloak}
\begin{enumerate}[label=\arabic*.]
    \item \textbf{Verbeterde schaalbaarheidsopties}: Implementeer verbeterde schaalbaarheidsopties om de prestaties te optimaliseren voor grootschalige implementaties.
    \item \textbf{Uitbreiding van integraties}: Werk aan het uitbreiden van integraties met een breder scala aan programmeertalen, frameworks en platforms.
    \item \textbf{Verbeterde documentatie}: Investeer in het verbeteren van de documentatie met uitgebreidere handleidingen, tutorials en voorbeelden.
\end{enumerate}

\subsubsection{Verbeteringen voor Hydra:}%
\label{subsubsec:verbeteringen-voor-hydra}
\begin{enumerate}[label=\arabic*.]
    \item \textbf{Uitbreiding van integraties}: Breid de integraties uit met populaire frameworks en platforms.
    \item \textbf{Verbeterde prestaties}: Werk aan het optimaliseren van de prestaties van Hydra, vooral voor grootschalige implementaties.
    \item \textbf{Beveiligingsverbeteringen}: Investeer in het verbeteren van de beveiliging van Hydra door regelmatige audits uit te voeren.
\end{enumerate}

\subsubsection{Verbeteringen voor Oathkeeper:}%
\label{subsubsec:verbeteringen-voor-oathkeeper}
\begin{enumerate}[label=\arabic*.]
    \item \textbf{Uitbreiding van functieset}: Voeg aanvullende functies toe, zoals ingebouwde ondersteuning voor Multi-Factor Authentication (MFA).
    \item \textbf{Gebruiksvriendelijkheid verbeteren}: Werk aan het verbeteren van de gebruikerservaring en gebruiksvriendelijkheid van het systeem.
    \item \textbf{Community-engagement vergroten}: Stimuleer community-engagement door het organiseren van evenementen en webinars.
\end{enumerate}

\subsection{Suggesties voor de keuze van een bestaand systeem:}%
\label{subsec:suggesties-voor-de-keuze-van-een-bestaand-systeem}
\begin{enumerate}[label=\arabic*.]
    \item \textbf{Keycloak}: Kies Keycloak als u een gebruiksvriendelijke en krachtige oplossing nodig heeft met uitgebreide documentatie en community-ondersteuning.
    \item \textbf{Hydra}: Selecteer Hydra als u behoefte heeft aan een zeer schaalbare en flexibele oplossing met geavanceerde autorisatiefuncties.
    \item \textbf{Oathkeeper}: Overweeg Oathkeeper als u een lichtgewicht en eenvoudig te configureren oplossing nodig heeft voor toegangscontrole in API-gateway-omgevingen.
\end{enumerate}