%%=============================================================================
%% Samenvatting
%%=============================================================================

% TODO: De "abstract" of samenvatting is een kernachtige (~ 1 blz. voor een
% thesis) synthese van het document.
%
% Een goede abstract biedt een kernachtig antwoord op volgende vragen:
%
% 1. Waarover gaat de bachelorproef?
% 2. Waarom heb je er over geschreven?
% 3. Hoe heb je het onderzoek uitgevoerd?
% 4. Wat waren de resultaten? Wat blijkt uit je onderzoek?
% 5. Wat betekenen je resultaten? Wat is de relevantie voor het werkveld?
%
% Daarom bestaat een abstract uit volgende componenten:
%
% - inleiding + kaderen thema
% - probleemstelling
% - (centrale) onderzoeksvraag
% - onderzoeksdoelstelling
% - methodologie
% - resultaten (beperk tot de belangrijkste, relevant voor de onderzoeksvraag)
% - conclusies, aanbevelingen, beperkingen
%
% LET OP! Een samenvatting is GEEN voorwoord!

%%---------- Nederlandse samenvatting -----------------------------------------
%
% TODO: Als je je bachelorproef in het Engels schrijft, moet je eerst een
% Nederlandse samenvatting invoegen. Haal daarvoor onderstaande code uit
% commentaar.
% Wie zijn bachelorproef in het Nederlands schrijft, kan dit negeren, de inhoud
% wordt niet in het document ingevoegd.

\IfLanguageName{english}{%
\selectlanguage{dutch}
\chapter*{Samenvatting}
\lipsum[1-4]
\selectlanguage{english}
}{}

%%---------- Samenvatting -----------------------------------------------------
% De samenvatting in de hoofdtaal van het document

\chapter*{\IfLanguageName{dutch}{Samenvatting}{Abstract}}

Dit onderzoeksvoorstel richt zich op de specifieke uitdagingen met betrekking tot authenticatie en autorisatie in applicatieontwikkeling, met als primaire 
focus het vereenvoudigen van deze processen. Het doel is om praktische oplossingen te identificeren en implementeren die relevant zijn voor ontwikkelaars 
die in hun projecten worden geconfronteerd met problemen op het gebied van beveiliging en gebruikerservaring. Dit aan de hand van een diepgaande analyse,
onderzoek en vergelijking van bestaande oplossingen/providers en technologieën/methoden, om zo de meest geschikte en effectieve aanpak te bepalen. Eventueel,
als er tijd over is, zal er een volledig authentificatie- en autorisatiesysteem worden ontwikkeld, dat als referentie kan dienen voor toekomstige
ontwikkelaars.
\newline
\newline
Het onderzoek wordt aangedreven door de aanvankelijke uitdagingen bij het beveiligen en toegankelijk maken van applicaties, met authenticatie en 
autorisatie als cruciale aspecten. Het richt zich op het bieden van een oplossing voor het beveiligen van diverse soorten apps, zoals desktop-, 
web- en mobiele apps. Graag wordt er een referentie-implementatie ontwikkeld, die ontwikkelaars kunnen gebruiken om hun eigen applicaties te beveiligen,
zonder te moeten af hangen van externe diensten of aanbieders. Vaak worden externe diensten gebruikt, zoals Auth0, TrustBuilder of Firebase, maar deze
kunnen beperkingen opleggen en kosten met zich meebrengen. Er wordt te vaak gekozen voor externe diensten, omdat dit vaak de eenvoudigste oplossing is.
Daarom wordt er in dit onderzoek onderzocht hoe men best te werk kan gaan om een eigen authentificatie- en autorisatiesysteem te ontwikkelen.
Opnieuw in geval dat er tijd over is, zal er een ontwikkeling plaatsvinden van een API-backend met diverse endpoints, waarmee ontwikkelaars functionaliteiten 
zoals inloggen, registreren, upgraden en meer kunnen implementeren. Daarnaast wordt gewerkt aan een gebruiksvriendelijk dashboard/frontend, waarmee 
ontwikkelaars hun gebruikers en andere relevante gegevens effectief kunnen beheren. 
\newline
\newline
Het onderzoek zal worden uitgevoerd door middel van een iteratieve aanpak, waarbij de focus ligt op het identificeren van de belangrijkste uitdagingen
en het ontwikkelen van oplossingen die deze uitdagingen aanpakken. Er zal een grondige analyse worden uitgevoerd van bestaande oplossingen en technologieën,
waarbij de voor- en nadelen van elke aanpak worden geëvalueerd. Op basis van deze analyse zal een aanpak worden gekozen en geïmplementeerd, waarbij
de nadruk ligt op het bieden van een eenvoudige, effectieve en veilige oplossing.
Er zal worden gezocht naar bestaande auth providers. Deze zullen worden geanalyseerd en vergeleken. Er zal worden gekeken naar de voor- en nadelen van
elke provider. Het zelfde geldt voor de technologieën en methoden die zullen worden gebruikt.
\newline
\newline
Uit het huidig onderzoekt blijkt dat er uiteraard al veel bestaande providers zijn, maar dat deze vaak beperkingen opleggen en kosten met zich meebrengen.
Door het opstellen van de vergelijkingscriteria, kan er een duidelijk beeld worden geschetst van de voor- en nadelen van elke provider. Dit zal helpen
om een weloverwogen keuze te maken. Na het onderzoeken van de bestaand methoden en technologieën, wordt er een duidelijk beeld geschetst van de mogelijkheden
en beperkingen van elke methode. Dit zal helpen om een weloverwogen keuze te maken bij de combinatie van methoden en technologieën die kunnen worden gebruikt.
\newline
\newline
De doelgroep van dit onderzoek zijn ontwikkelaars die specifieke uitdagingen ervaren bij het implementeren van effectieve authenticatie- en 
autorisatiesystemen. De focus ligt op concrete problemen waarmee ontwikkelaars worden geconfronteerd, zoals beveiligingslekken, complexe implementaties 
en gebruikerservaring problemen. 
\newline
\newline
De meerwaarde van dit onderzoek ligt in het bieden van praktische oplossingen voor ontwikkelaars die vergelijkbare uitdagingen tegenkomen. 
Dit onderzoek beoogt een volledig authentiek systeem te ontwikkelen (optioneel) en te documenteren. De meerwaarde ligt in het verschaffen van gedetailleerde 
inzichten in het ontwikkelproces, inclusief overwegingen bij de keuze van technologieën en de ervaringen tijdens de implementatie. Hierdoor kan het 
dienen als een waardevolle referentie en informatiebron voor toekomstige ontwikkelaars, waarbij het hen helpt om efficiënter en effectiever met 
soortgelijke uitdagingen om te gaan.
