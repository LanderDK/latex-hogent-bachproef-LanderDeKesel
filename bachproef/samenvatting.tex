%%=============================================================================
%% Samenvatting
%%=============================================================================

% TODO: De "abstract" of samenvatting is een kernachtige (~ 1 blz. voor een
% thesis) synthese van het document.
%
% Een goede abstract biedt een kernachtig antwoord op volgende vragen:
%
% 1. Waarover gaat de bachelorproef?
% 2. Waarom heb je er over geschreven?
% 3. Hoe heb je het onderzoek uitgevoerd?
% 4. Wat waren de resultaten? Wat blijkt uit je onderzoek?
% 5. Wat betekenen je resultaten? Wat is de relevantie voor het werkveld?
%
% Daarom bestaat een abstract uit volgende componenten:
%
% - inleiding + kaderen thema
% - probleemstelling
% - (centrale) onderzoeksvraag
% - onderzoeksdoelstelling
% - methodologie
% - resultaten (beperk tot de belangrijkste, relevant voor de onderzoeksvraag)
% - conclusies, aanbevelingen, beperkingen
%
% LET OP! Een samenvatting is GEEN voorwoord!

%%---------- Nederlandse samenvatting -----------------------------------------
%
% TODO: Als je je bachelorproef in het Engels schrijft, moet je eerst een
% Nederlandse samenvatting invoegen. Haal daarvoor onderstaande code uit
% commentaar.
% Wie zijn bachelorproef in het Nederlands schrijft, kan dit negeren, de inhoud
% wordt niet in het document ingevoegd.

\IfLanguageName{english}{%
\selectlanguage{dutch}
\chapter*{Samenvatting}
\lipsum[1-4]
\selectlanguage{english}
}{}

%%---------- Samenvatting -----------------------------------------------------
% De samenvatting in de hoofdtaal van het document

\chapter*{\IfLanguageName{dutch}{Samenvatting}{Abstract}}

Dit onderzoeksvoorstel richt zich op de specifieke uitdagingen met betrekking tot authenticatie en autorisatie in applicatieontwikkeling, met als primaire 
focus het vereenvoudigen van deze processen. Het doel is om praktische oplossingen te identificeren die relevant zijn voor ontwikkelaars 
die in hun projecten worden geconfronteerd met problemen op het gebied van beveiliging en gebruikerservaring. Dit aan de hand van een diepgaande analyse,
onderzoek en vergelijking van bestaande oplossingen/providers en technologieën/methoden, om zo de meest geschikte en effectieve aanpak te bepalen. 
\newline
\newline
Het onderzoek wordt aangedreven door de aanvankelijke uitdagingen bij het beveiligen en toegankelijk maken van applicaties, met authenticatie en 
autorisatie als cruciale aspecten. Het richt zich op het bieden van een oplossing voor het beveiligen van diverse soorten apps, zoals desktop-, 
web- en mobiele apps. Vandaag de dag zijn er al heel wat opties aanwezig hoe men authenticatie en autorisatie kan implementeren in hun applicaties.
Het doel is dus om te achterhalen hoe dit kan worden aangeboden aan elke ontwikkelaar.
\newline
\newline
Er zal een grondige analyse worden uitgevoerd van bestaande oplossingen en technologieën. Als eerst worden de bestaande technologieën onderzocht,
dit om een grondige kennis te verkrijgen over hoe deze functioneren en vooral hoe deze kunnen worden geïmplementeerd indien men kies een eigen
authenticatieserver te ontwikkelen. Vervolgens zullen bestaande oplossingen worden geanalyseerd om inzicht te krijgen wat de standaarden zijn die
men gebruikt bij de ontwikkeling van een authenticatieserver. 
Door het opstellen van de vergelijkingscriteria, kan er een duidelijk beeld worden geschetst van de voor- en nadelen van elke provider.
Maar vooral om te kijken of men al dan niet kan vaststellen dat dit hun antwoord is.
Indien men niet wil afhangen van een derde partij gaan we verder: de analyse of een lichtgewicht instantie, die bijvoorbeeld op een Docker image
kan draaien, bestaat of niet. Indien dit nog niet bestaat zal er een proof-of-concept worden ontwikkeld, de Docker image zal OAuth 2.0
autorisatie framework gaan implementeren. Indien dit wel al bestaat zal een vergelijkende studie bestaande alternatieven vergelijken en suggereren.
\newline
\newline
Uit het onderzoek word er veel kennis opgedaan omtrent de bestaande auth technologieën. Met de informatie die beschikbaar is door deze analyse
zou een ontwikkelaar een eigen authenticatieserver kunnen ontwikkelen. Vervolgens concluderen we uit het huidig onderzoekt dat er al veel bestaande 
providers zijn, maar dat deze vaak beperkingen opleggen en kosten met zich meebrengen. Als men graag zelf iets implementeert, maar niet volledig
van nul wilt beginnen, kan men er voor kiezen een lichtgewicht instantie te ontwikkelen. Door dit domein verder te onderzoeken, werd vastgesteld 
dat deze reeds bestonden. Daarom werd er vervolgens voor gekozen om een vergelijkende studie te houden over de bestaande alternatieven. Uit deze
studie bleek het al snel dat er maar drie overbleven: Keycloak, Hydra en Oathkeeper. Het resultaat van dit onderzoek stelt dus vast dat men een bestaande
lichtgewicht instantie kan implementeren, zoals Keycloak, als men volledige controle wil. Het tweede alternatief is een bestaande oplossing/provider
gebruiken, houd er wel rekening mee dat dit beperkingen en kosten kan met zich meebrengen. Het derde en laatste alternatief, dat vaak het minst zal
gekozen worden, is om zelf een authenticatieserver te ontwikkelen die bijvoorbeeld het OAuth 2.0 autorisatie framework implementeert. Dit laatste alternatief
zal de meeste inspanning kosten, maar zal ongetwijfeld de meeste kennis en ervaring opleveren.
\newline
\newline
Conclusie: authenticatie en autorisatie in applicatieontwikkeling is al veel verder ontwikkeld dan initiëel gedacht bij het begin van dit onderzoek.
Dit onderzoek werd gestart met weinig kennis over auth en werd beëindigd met een grondige basis om zelf een authenticatieserver te ontwikkelen.
Maar dit is niet de echte conclusie van dit onderzoek. Als een ontwikkelaar of een bedrijf authenticatie en autorisatie in hun applicaties wilt
integreren, kunnen we concluderen dat er vaak zal gegrepen worden naar een bestaande provider. Dit is de makkelijkste weg, maar vaak niet de
goedkoopste weg. Vaak denken bedrijven alleen aan de korte termijn voordelen en zal er vaak op lange termijn worden ingezien dat er een andere
mogelijkheid was. Zoals een Docker image van Keycloak bijvoorbeeld, die on-premise draait. Of men kan nog verder gaan: een zelf ontwikkelde
authenticatieserver implementeren die het OAuth 2.0 autorisatie framework implementeert. Wat gesuggereerd wordt: bij weinig ervaring gebruikt men
best een bestaande provider, bij matige ervaring kan men Keycloak on-premise gebruiken, en als laatste bij voldoende ervaring kan men een zelf
ontwikkelde authenticatieserver implementeren die het OAuth 2.0 autorisatie framework implementeert.
