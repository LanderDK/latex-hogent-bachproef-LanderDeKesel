%%=============================================================================
%% Conclusie
%%=============================================================================

\chapter{Conclusie}%
\label{ch:conclusie}

% TODO: Trek een duidelijke conclusie, in de vorm van een antwoord op de
% onderzoeksvra(a)g(en). Wat was jouw bijdrage aan het onderzoeksdomein en
% hoe biedt dit meerwaarde aan het vakgebied/doelgroep? 
% Reflecteer kritisch over het resultaat. In Engelse teksten wordt deze sectie
% ``Discussion'' genoemd. Had je deze uitkomst verwacht? Zijn er zaken die nog
% niet duidelijk zijn?
% Heeft het onderzoek geleid tot nieuwe vragen die uitnodigen tot verder 
%onderzoek?

Dit onderzoek heeft zich gericht op het ondersteunen van ontwikkelaars bij het implementeren van effectieve en veilige authenticatie- en autorisatie 
systemen in applicaties. Het doel was om een grondige analyse te maken van bestaande oplossingen en technologieën, en om verbeteringen of aanpassingen 
voor te stellen om de uitdagingen en problemen die ontwikkelaars tegenkomen aan te pakken. 
\\\\
Uit het onderzoek blijkt dat er al veel bestaande providers zijn, maar dat deze vaak beperkingen opleggen en kosten met zich meebrengen. Als alternatief
kan men ervoor kiezen om een lichtgewicht instantie te ontwikkelen of zelf een authenticatieserver te ontwikkelen die het OAuth 2.0 autorisatie framework
implementeert.
\\\\
Het onderzoek heeft aangetoond dat authenticatie en autorisatie in applicatieontwikkeling al veel verder ontwikkeld is dan aanvankelijk gedacht.
Ontwikkelaars of bedrijven die authenticatie en autorisatie in hun applicaties willen integreren, zullen vaak kiezen voor een bestaande provider.
Dit is de makkelijkste weg, maar vaak niet de goedkoopste. Een alternatief zou kunnen zijn om een Docker image van Keycloak on-premise te gebruiken,
of om zelf een authenticatieserver te ontwikkelen die het OAuth 2.0 autorisatie framework implementeert.
\\\\
Het onderzoek heeft geleid tot nieuwe vragen die uitnodigen tot verder onderzoek. Bijvoorbeeld, hoe kunnen we de gebruikerservaring verbeteren bij
het gebruik van een zelf ontwikkelde authenticatieserver? Of, hoe kunnen we de kosten verlagen bij het gebruik van een bestaande provider? Deze vragen
kunnen het onderwerp zijn van toekomstig onderzoek in dit domein.
\\\\
De bijdrage van dit onderzoek aan het vakgebied is het bieden van een dieper inzicht in de uitdagingen en mogelijke oplossingen voor het implementeren
van authenticatie en autorisatie in applicaties. Het biedt ontwikkelaars een startpunt voor het maken van geïnformeerde beslissingen over welke oplossing
het beste past bij hun specifieke behoeften en omstandigheden.
