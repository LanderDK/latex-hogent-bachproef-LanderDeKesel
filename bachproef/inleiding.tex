%%=============================================================================
%% Inleiding
%%=============================================================================

\chapter{\IfLanguageName{dutch}{Inleiding}{Introduction}}%
\label{ch:inleiding}

In de hedendaagse digitale wereld zijn applicaties niet meer weg te denken uit ons dagelijks leven. Of het nu gaat om desktop-, web- of mobiele applicaties, ze vormen de ruggengraat van moderne interactie 
en dienstverlening. Echter, met de toename van digitale transacties en gegevensuitwisseling, worden veiligheidsaspecten zoals authenticatie en autorisatie steeds crucialer. Het beveiligen van deze applicaties 
terwijl een naadloze gebruikerservaring behouden blijft, vormt een uitdagende taak voor ontwikkelaars.
\newline
\newline
Dit onderzoeksvoorstel richt zich op de specifieke uitdagingen met betrekking tot authenticatie en autorisatie in applicatieontwikkeling. Het beoogt deze processen te vereenvoudigen en praktische oplossingen 
te identificeren en implementeren die relevant zijn voor ontwikkelaars die met beveiliging- en gebruikerservaringsproblemen worden geconfronteerd. Hierbij wordt een diepgaande analyse, onderzoek en 
vergelijking van bestaande oplossingen en technologieën uitgevoerd om de meest geschikte en effectieve aanpak te bepalen.
\newline
\newline
Het onderzoek wordt aangedreven door de initiële uitdagingen bij het beveiligen en toegankelijk maken van applicaties, waarbij authenticatie en autorisatie als cruciale aspecten worden beschouwd. 
Er wordt gestreefd naar het bieden van een referentie-implementatie die ontwikkelaars kunnen gebruiken om hun eigen applicaties te beveiligen, zonder afhankelijk te zijn van externe diensten of aanbieders die vaak beperkingen opleggen en kosten met zich meebrengen.
\newline
\newline
Door middel van een iteratieve aanpak zal dit onderzoek zich richten op het identificeren van de belangrijkste uitdagingen en het ontwikkelen van oplossingen die deze uitdagingen aanpakken. 
Hierbij zal een grondige analyse worden uitgevoerd van bestaande oplossingen en technologieën, waarbij de nadruk ligt op het bieden van een eenvoudige, effectieve en veilige oplossing.
\newline
\newline
Het doel van dit onderzoek is om ontwikkelaars die specifieke uitdagingen ervaren bij het implementeren van effectieve authenticatie- en autorisatiesystemen, te ondersteunen en richtlijnen te bieden om deze uitdagingen te overwinnen.
De focus ligt op concrete problemen waarmee ontwikkelaars worden geconfronteerd, zoals beveiligingslekken, complexe implementaties en gebruikerservaring problemen.


\section{\IfLanguageName{dutch}{Probleemstelling}{Problem Statement}}%
\label{sec:probleemstelling}
In de huidige digitale wereld is de beveiliging van applicaties een cruciaal aspect geworden. Authenticatie en autorisatie zijn twee belangrijke veiligheidsaspecten die een uitdaging vormen voor ontwikkelaars. 
Het implementeren van deze aspecten op een manier die zowel veilig als gebruiksvriendelijk is, is vaak complex en tijdrovend. Bovendien kunnen fouten in de implementatie leiden tot ernstige beveiligingslekken. 
\newline
Daarnaast zijn veel bestaande oplossingen afhankelijk van externe diensten of aanbieders, die vaak beperkingen opleggen en kosten met zich meebrengen. Dit kan een belemmering vormen voor ontwikkelaars, 
vooral voor diegenen die werken aan kleinere projecten of met beperkte budgetten.
\newline
Het doel van dit onderzoek is om bestaande oplossingen en technologieën voor authenticatie en autorisatie in applicatieontwikkeling te analyseren, vergelijken en evalueren. 
Het richt zich op het identificeren van pijnpunten en het doen van suggesties om ontwikkelaars een duidelijke richtlijn te bieden voor het implementeren van veilige applicaties. 
Het doel is om ontwikkelaars te ondersteunen bij het overwinnen van uitdagingen op het gebied van beveiliging en gebruikerservaring.

\section{\IfLanguageName{dutch}{Onderzoeksvraag}{Research question}}%
\label{sec:onderzoeksvraag}

De centrale onderzoeksvraag van dit onderzoek is: ``Hoe kunnen ontwikkelaars effectieve en veilige authenticatie- en autorisatiesystemen implementeren in applicaties, zonder afhankelijk te zijn van 
externe diensten of aanbieders, en tegelijkertijd een optimale gebruikerservaring bieden?''
\newline
\newline
Deze centrale vraag kan verder worden opgesplitst in de volgende deelvragen:
\newline
\newline
1. Wat zijn de huidige uitdagingen en problemen bij het implementeren van authenticatie en autorisatie in applicaties?
\newline
2. Welke bestaande oplossingen en technologieën zijn er momenteel beschikbaar voor het implementeren van authenticatie en autorisatie in applicaties, en wat zijn hun voor- en nadelen?
\newline
3. Hoe kunnen deze oplossingen en technologieën worden verbeterd of aangepast om de uitdagingen en problemen aan te pakken die ontwikkelaars tegenkomen?
\newline
4. Hoe kunnen deze verbeterde of aangepaste oplossingen en technologieën worden geïmplementeerd in een referentie-implementatie die kan dienen als een praktische gids voor ontwikkelaars?

\section{\IfLanguageName{dutch}{Onderzoeksdoelstelling}{Research objective}}%
\label{sec:onderzoeksdoelstelling}

Het beoogde resultaat van dit onderzoek is een grondige analyse van bestaande oplossingen en technologieën voor authenticatie en autorisatie in applicatieontwikkeling, en het voorstellen van 
verbeteringen of aanpassingen om de uitdagingen en problemen die ontwikkelaars tegenkomen aan te pakken. 
\newline
\newline
Het succes van dit onderzoek zal worden gemeten aan de hand van de volgende criteria:
\newline
\newline
1. Het identificeren van de belangrijkste uitdagingen en problemen bij het implementeren van authenticatie en autorisatie in applicaties.
\newline
2. Het uitvoeren van een grondige analyse van bestaande oplossingen en technologieën, en het identificeren van hun voor- en nadelen.
\newline
3. Het voorstellen van verbeteringen of aanpassingen aan deze oplossingen en technologieën om de geïdentificeerde uitdagingen en problemen aan te pakken.
\newline
4. Het opstellen van een verslag met aanbevelingen op basis van een vergelijkende studie van de geanalyseerde oplossingen en technologieën.
\newline
\newline
Het uiteindelijke doel is om ontwikkelaars te ondersteunen bij het overwinnen van uitdagingen op het gebied van beveiliging en gebruikerservaring, en hen te helpen bij het bouwen van veilige en 
gebruiksvriendelijke applicaties.

\section{\IfLanguageName{dutch}{Opzet van deze bachelorproef}{Structure of this bachelor thesis}}%
\label{sec:opzet-bachelorproef}

% Het is gebruikelijk aan het einde van de inleiding een overzicht te
% geven van de opbouw van de rest van de tekst. Deze sectie bevat al een aanzet
% die je kan aanvullen/aanpassen in functie van je eigen tekst.

De rest van deze bachelorproef is als volgt opgebouwd:

In Hoofdstuk~\ref{ch:stand-van-zaken} wordt een overzicht gegeven van de stand van zaken binnen het onderzoeksdomein, op basis van een literatuurstudie.

In Hoofdstuk~\ref{ch:methodologie} wordt de methodologie toegelicht en worden de gebruikte onderzoekstechnieken besproken om een antwoord te kunnen formuleren op de onderzoeksvragen.

% TODO: Vul hier aan voor je eigen hoofdstukken, één of twee zinnen per hoofdstuk

In Hoofdstuk~\ref{ch:conclusie}, tenslotte, wordt de conclusie gegeven en een antwoord geformuleerd op de onderzoeksvragen. Daarbij wordt ook een aanzet gegeven voor toekomstig onderzoek binnen dit domein.