%%=============================================================================
%% Inleiding
%%=============================================================================

\chapter{\IfLanguageName{dutch}{Inleiding}{Introduction}}%
\label{ch:inleiding}

In de hedendaagse digitale wereld zijn applicaties niet meer weg te denken uit ons dagelijks leven. Of het nu gaat om desktop-, web- of mobiele applicaties, ze vormen de ruggengraat van moderne interactie en dienstverlening. Echter, met de toename van digitale transacties en gegevensuitwisseling, worden veiligheidsaspecten zoals authenticatie en autorisatie steeds crucialer. Het beveiligen van deze applicaties terwijl een naadloze gebruikerservaring behouden blijft, vormt een uitdagende taak voor ontwikkelaars.
\newline
\newline
Dit onderzoeksvoorstel richt zich op de specifieke uitdagingen met betrekking tot authenticatie en autorisatie in applicatieontwikkeling. Het beoogt deze processen te vereenvoudigen en praktische oplossingen te identificeren en implementeren die relevant zijn voor ontwikkelaars die met beveiligings- en gebruikerservaringsproblemen worden geconfronteerd. Hierbij wordt een diepgaande analyse, onderzoek en vergelijking van bestaande oplossingen en technologieën uitgevoerd om de meest geschikte en effectieve aanpak te bepalen.
\newline
\newline
Het onderzoek wordt aangedreven door de initiële uitdagingen bij het beveiligen en toegankelijk maken van applicaties, waarbij authenticatie en autorisatie als cruciale aspecten worden beschouwd. Er wordt gestreefd naar het bieden van een referentie-implementatie die ontwikkelaars kunnen gebruiken om hun eigen applicaties te beveiligen, zonder afhankelijk te zijn van externe diensten of aanbieders die vaak beperkingen opleggen en kosten met zich meebrengen.
\newline
\newline
Door middel van een iteratieve aanpak zal dit onderzoek zich richten op het identificeren van de belangrijkste uitdagingen en het ontwikkelen van oplossingen die deze uitdagingen aanpakken. Hierbij zal een grondige analyse worden uitgevoerd van bestaande oplossingen en technologieën, waarbij de nadruk ligt op het bieden van een eenvoudige, effectieve en veilige oplossing.
\newline
\newline
De doelgroep van dit onderzoek zijn ontwikkelaars die specifieke uitdagingen ervaren bij het implementeren van effectieve authenticatie- en autorisatiesystemen. De focus ligt op concrete problemen waarmee ontwikkelaars worden geconfronteerd, zoals beveiligingslekken, complexe implementaties en gebruikerservaring problemen.
\newline
\newline
De meerwaarde van dit onderzoek ligt in het bieden van praktische oplossingen voor ontwikkelaars die vergelijkbare uitdagingen tegenkomen. Dit onderzoek beoogt een volledig authentiek systeem te ontwikkelen en te documenteren, waardoor het kan dienen als een waardevolle referentie en informatiebron voor toekomstige ontwikkelaars om efficiënter en effectiever met soortgelijke uitdagingen om te gaan.

\section{\IfLanguageName{dutch}{Probleemstelling}{Problem Statement}}%
\label{sec:probleemstelling}

Uit je probleemstelling moet duidelijk zijn dat je onderzoek een meerwaarde heeft voor een concrete doelgroep. De doelgroep moet goed gedefinieerd en afgelijnd zijn. Doelgroepen als ``bedrijven,'' ``KMO's'', systeembeheerders, enz.~zijn nog te vaag. Als je een lijstje kan maken van de personen/organisaties die een meerwaarde zullen vinden in deze bachelorproef (dit is eigenlijk je steekproefkader), dan is dat een indicatie dat de doelgroep goed gedefinieerd is. Dit kan een enkel bedrijf zijn of zelfs één persoon (je co-promotor/opdrachtgever).

\section{\IfLanguageName{dutch}{Onderzoeksvraag}{Research question}}%
\label{sec:onderzoeksvraag}

Wees zo concreet mogelijk bij het formuleren van je onderzoeksvraag. Een onderzoeksvraag is trouwens iets waar nog niemand op dit moment een antwoord heeft (voor zover je kan nagaan). Het opzoeken van bestaande informatie (bv. ``welke tools bestaan er voor deze toepassing?'') is dus geen onderzoeksvraag. Je kan de onderzoeksvraag verder specifiëren in deelvragen. Bv.~als je onderzoek gaat over performantiemetingen, dan 

\section{\IfLanguageName{dutch}{Onderzoeksdoelstelling}{Research objective}}%
\label{sec:onderzoeksdoelstelling}

Wat is het beoogde resultaat van je bachelorproef? Wat zijn de criteria voor succes? Beschrijf die zo concreet mogelijk. Gaat het bv.\ om een proof-of-concept, een prototype, een verslag met aanbevelingen, een vergelijkende studie, enz.

\section{\IfLanguageName{dutch}{Opzet van deze bachelorproef}{Structure of this bachelor thesis}}%
\label{sec:opzet-bachelorproef}

% Het is gebruikelijk aan het einde van de inleiding een overzicht te
% geven van de opbouw van de rest van de tekst. Deze sectie bevat al een aanzet
% die je kan aanvullen/aanpassen in functie van je eigen tekst.

De rest van deze bachelorproef is als volgt opgebouwd:

In Hoofdstuk~\ref{ch:stand-van-zaken} wordt een overzicht gegeven van de stand van zaken binnen het onderzoeksdomein, op basis van een literatuurstudie.

In Hoofdstuk~\ref{ch:methodologie} wordt de methodologie toegelicht en worden de gebruikte onderzoekstechnieken besproken om een antwoord te kunnen formuleren op de onderzoeksvragen.

% TODO: Vul hier aan voor je eigen hoofstukken, één of twee zinnen per hoofdstuk

In Hoofdstuk~\ref{ch:conclusie}, tenslotte, wordt de conclusie gegeven en een antwoord geformuleerd op de onderzoeksvragen. Daarbij wordt ook een aanzet gegeven voor toekomstig onderzoek binnen dit domein.