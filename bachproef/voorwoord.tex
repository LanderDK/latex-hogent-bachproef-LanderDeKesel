%%=============================================================================
%% Voorwoord
%%=============================================================================

\chapter*{\IfLanguageName{dutch}{Woord vooraf}{Preface}}%
\label{ch:voorwoord}

%% TODO:
%% Het voorwoord is het enige deel van de bachelorproef waar je vanuit je
%% eigen standpunt (``ik-vorm'') mag schrijven. Je kan hier bv. motiveren
%% waarom jij het onderwerp wil bespreken.
%% Vergeet ook niet te bedanken wie je geholpen/gesteund/... heeft

Het schrijven van deze bachelorproef was een uitdagende maar verrijkende ervaring. Het onderwerp van authenticatie en autorisatie in applicatieontwikkeling heeft altijd mijn interesse gewekt, 
vooral vanwege de cruciale rol die het speelt in de huidige digitale wereld. Het was een kans om dieper in dit onderwerp te duiken en de complexiteit ervan te begrijpen.
Ik heb dit onderzoek gekozen omdat ik al sinds het begin van mijn opleiding, en zelfs daarvoor, geïnteresseerd ben in beveiliging en privacy. Het is een onderwerp dat steeds belangrijker wordt 
in de digitale wereld, en ik wilde graag meer leren over de verschillende aspecten ervan. Dit onderzoek heeft me geholpen om een beter begrip te krijgen van de uitdagingen en oplossingen op het gebied van
authenticatie en autorisatie in applicatieontwikkeling.
\newline
Ik ben al altijd bezig geweest met het beveiligen, of ten minste poging tot beveiligen, van mijn eigen applicaties en websites. Vaak gebruikte ik bestaande oplossingen/aanbieders, maar ik heb altijd al ``meer'' gewild.
Ik wilde meer weten over hoe deze systemen werken, meer weten over hoe ik dit ooit zelf zou kunnen implementeren. Daarom ben ik, een jaar voor dit onderzoek, begonnen met informatie te verzamelen en
het effectief ontwikkelen van een eigen authenticatie- en autorisatiesysteem. Maar ik wist dat er iets miste, ik wist dat er nog veel meer te leren was.
\newline
Daarom ben ik begonnen met dit onderzoek. Ik zag dit als de ultieme kans om meer te leren over dit onderwerp, en om mijn kennis te delen met anderen. En ik ben blij dat ik dat gedaan heb.
Ik ben vrij zeker dat ik nu zo goed als alle kennis en informatie heb verzameld die ik nodig heb om een veilig en betrouwbaar authenticatie- en autorisatiesysteem te ontwikkelen. Ook hoop ik dat dit
onderzoek anderen kan helpen met hetzelfde doel, of toch ten minste een beter begrip kan geven van de complexiteit van dit onderwerp.
\newline
\newline
Ik wil graag mijn promotor bedanken voor zijn waardevolle feedback en begeleiding gedurende dit proces. Zijn inzichten en suggesties hebben enorm bijgedragen aan de kwaliteit van dit werk. 
\newline
\newline
Ik hoop dat dit werk nuttig zal zijn voor andere ontwikkelaars die geconfronteerd worden met de uitdagingen van authenticatie en autorisatie in applicatieontwikkeling.

\begin{flushright}
Lander De Kesel
\end{flushright}